

\beginsubsection{Overview of The Lisp Printer}


\clisp\ provides a representation of most //[[CL:Glossary:objects]]// in the form 
of printed text called the printed representation.
Functions such as **[[CL:Functions:print]]** take an //[[CL:Glossary:object]]// 
and send the characters of its printed representation to a //[[CL:Glossary:stream]]//. 
The collection of routines that does this is known as the (\clisp) printer.  


Reading a printed representation 


typically
produces an //[[CL:Glossary:object]]// that is **[[CL:Functions:equal]]** to the
originally printed //[[CL:Glossary:object]]//.

\beginsubsubsection{Multiple Possible Textual Representations}


Most //[[CL:Glossary:objects]]// have more than one possible textual representation.
For example, the positive //[[CL:Glossary:integer]]// with a magnitude of twenty-seven
can be textually expressed in any of these ways:

\code
 27    27.    #o33    #x1B    #b11011    #.(* 3 3 3)    81/3
\endcode

A list containing the two symbols \f{A} and \f{B} can also be textually
expressed in a variety of ways:

\code
 (A B)    (a b)    (  a  b )    (\\A |B|) 
(|\\A|
  B
)
\endcode

In general,
\issue{PRINTER-WHITESPACE:JUST-ONE-SPACE}



from the point of view of the //[[CL:Glossary:Lisp reader]]//,

wherever //[[CL:Glossary:whitespace]]// is permissible in a textual representation,
any number of //[[CL:Glossary:spaces]]// and //[[CL:Glossary:newlines]]// can appear in //[[CL:Glossary:standard syntax]]//.


When a function such as **[[CL:Functions:print]]** produces a printed representation,
it must choose 


from among many possible textual representations.
In most cases, it chooses a 


program readable representation,
but in certain cases it might use a more compact notation that is not 
program-readable.

A number of option variables, called
\newtermidx{printer control variables}{printer control variable},
are provided to permit control of individual aspects of the 
printed representation of //[[CL:Glossary:objects]]//.
\Thenextfigure\ shows the //[[CL:Glossary:standardized]]// //[[CL:Glossary:printer control variables]]//;
there might also be //[[CL:Glossary:implementation-defined]]// //[[CL:Glossary:printer control variables]]//.

\DefineFigure{StdPrinterControlVars}
\displaythree{Standardized Printer Control Variables}{
*print-array*&*print-gensym*&*print-pprint-dispatch*\cr
*print-base*&*print-length*&*print-pretty*\cr
*print-case*&*print-level*&*print-radix*\cr
*print-circle*&*print-lines*&*print-readably*\cr
*print-escape*&*print-miser-width*&*print-right-margin*\cr
}

In addition to the //[[CL:Glossary:printer control variables]]//, 
the following additional //[[CL:Glossary:defined names]]// 
relate to or affect the behavior of the //[[CL:Glossary:Lisp printer]]//:

\displaythree{Additional Influences on the Lisp printer.}{
*package*&*read-eval*&readtable-case\cr
*read-default-float-format*&*readtable*&\cr
}

\beginsubsubsubsection{Printer Escaping}

\Thevariable{*print-escape*} controls whether the //[[CL:Glossary:Lisp printer]]//
tries to produce notations such as escape characters and package prefixes.

\Thevariable{*print-readably*} can be used to override
many of the individual aspects controlled by the other 
//[[CL:Glossary:printer control variables]]// when program-readable output
is especially important.

\issue{PRINT-READABLY-BEHAVIOR:CLARIFY}
One of the many effects of making \thevalueof{*print-readably*} be //[[CL:Glossary:true]]//
is that the //[[CL:Glossary:Lisp printer]]// behaves as if \varref{*print-escape*} were also //[[CL:Glossary:true]]//.
For notational convenience, we say that 
if the value of either \varref{*print-readably*} or \varref{*print-escape*} is //[[CL:Glossary:true]]//, 
then //[[CL:Glossary:printer escaping]]// is ``enabled'';
and we say that
if the values of both \varref{*print-readably*} and \varref{*print-escape*} are //[[CL:Glossary:false]]//, 
then //[[CL:Glossary:printer escaping]]// is ``disabled''.


\endsubsubsubsection%{Printer Escaping}

\endsubsubsection%{Multiple Possible Textual Representations}

\endsubsection%{Overview of The Lisp Printer}

\beginsubsection{Printer Dispatching}
\DefineSection{PrinterDispatch}

\issue{DEFSTRUCT-PRINT-FUNCTION-AGAIN:X3J13-MAR-93}























The //[[CL:Glossary:Lisp printer]]// makes its determination of how to print an
//[[CL:Glossary:object]]// as follows:

If \thevalueof{*print-pretty*} is //[[CL:Glossary:true]]//, 
printing is controlled by the //[[CL:Glossary:current pprint dispatch table]]//;

\seesection\PPrintDispatchTables.

Otherwise (if \thevalueof{*print-pretty*} is //[[CL:Glossary:false]]//),
the object's **[[CL:Functions:print-object]]** method is used;
\seesection\DefaultPrintObjMeths.


\endsubsection%{Printer Dispatching}

\beginsubsection{Default Print-Object Methods}
\DefineSection{DefaultPrintObjMeths}




This section describes the default behavior of 
**[[CL:Functions:print-object]]** methods for the //[[CL:Glossary:standardized]]// //[[CL:Glossary:types]]//.

\beginsubsubsection{Printing Numbers}

\beginsubsubsubsection{Printing Integers}
\DefineSection{PrintingIntegers}


//[[CL:Glossary:Integers]]// are printed in the radix specified by the //[[CL:Glossary:current output base]]//
in positional notation, most significant digit first.
If appropriate, a radix specifier can be printed; see \varref{*print-radix*}.
If an //[[CL:Glossary:integer]]// is negative, a minus sign is printed and then the
absolute value of the //[[CL:Glossary:integer]]// is printed.
The //[[CL:Glossary:integer]]// zero is represented
by the single digit \f{0} and never has a sign.
A decimal point might be printed, 
depending on \thevalueof{*print-radix*}.

For related information about the syntax of an //[[CL:Glossary:integer]]//,
\seesection\SyntaxOfIntegers.

\endsubsubsubsection%{Printing Integers}
\beginsubsubsubsection{Printing Ratios}
\DefineSection{PrintingRatios}\idxref{ratio}


//[[CL:Glossary:Ratios]]// are printed as follows:
the absolute value of the numerator is printed, as for an //[[CL:Glossary:integer]]//;
then a \f{/}; then the denominator.  The numerator and denominator are
both printed in the radix specified by the //[[CL:Glossary:current output base]]//; 
they are obtained as if by
**[[CL:Functions:numerator]]** and **[[CL:Functions:denominator]]**, and so //[[CL:Glossary:ratios]]//
are printed in reduced form (lowest terms).
If appropriate, a radix specifier can be printed; see 
\varref{*print-radix*}.
If the ratio is negative, a minus sign is printed before the numerator.

For related information about the syntax of a //[[CL:Glossary:ratio]]//,
\seesection\SyntaxOfRatios.

\endsubsubsubsection%{Printing Ratios}
\beginsubsubsubsection{Printing Floats}
\DefineSection{PrintingFloats}\idxref{float}


If the magnitude of the //[[CL:Glossary:float]]// is either zero or between $10^{-3}$ (inclusive)
and $10^7$ (exclusive), it is printed as the integer part of the number,
then a decimal point,
followed by the fractional part of the number;
there is always at least one
digit on each side of the decimal point.    
If the sign of the number
(as determined by **[[CL:Functions:float-sign]]**)
is negative, then a minus sign is printed before the number.
If the format of the number
does not match that specified by
\varref{*read-default-float-format*}, then the //[[CL:Glossary:exponent marker]]// for
that format and the digit \f{0} are also printed.
For example, the base of the natural logarithms as a //[[CL:Glossary:short float]]//
might be printed as \f{2.71828S0}.


For non-zero magnitudes outside of the range $10^{-3}$ to $10^7$,
a //[[CL:Glossary:float]]// is printed in computerized scientific notation.
The representation of the number is scaled to be between
1 (inclusive) and 10 (exclusive) and then printed, with one digit
before the decimal point and at least one digit after the decimal point.
Next the //[[CL:Glossary:exponent marker]]// for the format is printed,
except that
if the format of the number matches that specified by 
\varref{*read-default-float-format*}, then the //[[CL:Glossary:exponent marker]]// \f{E}
is used.
Finally, the power of ten by which the fraction must be multiplied
to equal the original number is printed as a decimal integer.
For example, Avogadro's number as a //[[CL:Glossary:short float]]// 
is printed as \f{6.02S23}.

For related information about the syntax of a //[[CL:Glossary:float]]//,
\seesection\SyntaxOfFloats.

\endsubsubsubsection%{Printing Floats}
\beginsubsubsubsection{Printing Complexes}
\DefineSection{PrintingComplexes}\idxref{complex}


A //[[CL:Glossary:complex]]// is printed as \f{\#C}, an open parenthesis,
the printed representation of its real part, a space,
the printed representation of its imaginary part, and finally
a close parenthesis.

For related information about the syntax of a //[[CL:Glossary:complex]]//,
\seesection\SyntaxOfComplexes\ and \secref\SharpsignC.

\endsubsubsubsection%{Printing Complexes}
\beginsubsubsubsection{Note about Printing Numbers}

The printed representation of a number must not contain //[[CL:Glossary:escape]]// //[[CL:Glossary:characters]]//;
\seesection\EscCharsAndPotentialNums.

\endsubsubsubsection%{Note about Printing Numbers}
\endsubsubsection%{Printing Numbers}
\beginsubsubsection{Printing Characters}
\DefineSection{PrintingCharacters}


\issue{PRINT-READABLY-BEHAVIOR:CLARIFY}

When //[[CL:Glossary:printer escaping]]// is disabled,

a //[[CL:Glossary:character]]// prints as itself;
it is sent directly to the output //[[CL:Glossary:stream]]//.
\issue{PRINT-READABLY-BEHAVIOR:CLARIFY}

When //[[CL:Glossary:printer escaping]]// is enabled,

then \f{\#\\} syntax is used.



When the printer types out the name of a //[[CL:Glossary:character]]//,
it uses the same table as the \f{\#\\} //[[CL:Glossary:reader macro]]// would use;
therefore any //[[CL:Glossary:character]]// name that is typed out
is acceptable as input (in that //[[CL:Glossary:implementation]]//).
If a //[[CL:Glossary:non-graphic]]// //[[CL:Glossary:character]]// has a //[[CL:Glossary:standardized]]// //[[CL:Glossary:name]]//\meaning{5},
that //[[CL:Glossary:name]]// is preferred over non-standard //[[CL:Glossary:names]]//
for printing in \f{\#\\} notation.
For the //[[CL:Glossary:graphic]]// //[[CL:Glossary:standard characters]]//,
the //[[CL:Glossary:character]]// itself is always used
for printing in \f{\#\\} notation---even if 
the //[[CL:Glossary:character]]// also has a //[[CL:Glossary:name]]//\meaning{5}.

For details about the \f{\#\\} //[[CL:Glossary:reader macro]]//, \seesection\SharpsignBackslash.

\endsubsubsection%{Printing Characters}
\beginsubsubsection{Printing Symbols}
\DefineSection{PrintingSymbols}



\issue{PRINT-READABLY-BEHAVIOR:CLARIFY}

When //[[CL:Glossary:printer escaping]]// is disabled,

only the characters of the //[[CL:Glossary:symbol]]//'s //[[CL:Glossary:name]]// are output 

\issue{PRINT-CASE-BEHAVIOR:CLARIFY}


(but the case in which to print characters in the //[[CL:Glossary:name]]// is
controlled by \varref{*print-case*};
\seesection\ReadtableCasePrintEffect).



The remainder of \thissection\ applies only 
\issue{PRINT-READABLY-BEHAVIOR:CLARIFY}

when //[[CL:Glossary:printer escaping]]// is enabled.



\issue{SYMBOL-PRINT-ESCAPE-BEHAVIOR:CLARIFY}







When printing a //[[CL:Glossary:symbol]]//, the printer inserts enough 
//[[CL:Glossary:single escape]]// and/or //[[CL:Glossary:multiple escape]]//
characters (//[[CL:Glossary:backslashes]]// and/or //[[CL:Glossary:vertical-bars]]//) so that if
**[[CL:Functions:read]]** were called with the same \varref{*readtable*} and
with \varref{*read-base*} bound to the //[[CL:Glossary:current output base]]//, it
would return the same //[[CL:Glossary:symbol]]// (if it is not 
//[[CL:Glossary:apparently uninterned]]//) or an //[[CL:Glossary:uninterned]]// //[[CL:Glossary:symbol]]//
with the same //[[CL:Glossary:print name]]// (otherwise).

For example, if \thevalueof{*print-base*} were \f{16} 
when printing the symbol \f{face}, it would have to be printed as
\f{\\FACE} or \f{\\Face} or \f{|FACE|}, 
because the token \f{face} would be read as a hexadecimal
number (decimal value 64206) if \thevalueof{*read-base*} were \f{16}.

For additional restrictions concerning characters with  nonstandard
//[[CL:Glossary:syntax types]]// in the //[[CL:Glossary:current readtable]]//, \seevar{*print-readably*} 


For information about how the //[[CL:Glossary:Lisp reader]]// parses //[[CL:Glossary:symbols]]//,
\seesection\SymbolTokens\ and \secref\SharpsignColon.







\nil\ might be printed as \f{()} 
\issue{PRINT-READABLY-BEHAVIOR:CLARIFY}

when \varref{*print-pretty*} is //[[CL:Glossary:true]]//
and //[[CL:Glossary:printer escaping]]// is enabled.


\beginsubsubsubsection{Package Prefixes for Symbols}



//[[CL:Glossary:Package prefixes]]// are printed if necessary.
The rules for //[[CL:Glossary:package prefixes]]// are as follows.
When the //[[CL:Glossary:symbol]]// is printed, if it is in \thepackage{keyword}, 

then it is printed with a preceding //[[CL:Glossary:colon]]//; otherwise, if
it is //[[CL:Glossary:accessible]]// in the //[[CL:Glossary:current package]]//, it is printed without any
//[[CL:Glossary:package prefix]]//; otherwise, it is printed with a //[[CL:Glossary:package prefix]]//.


A //[[CL:Glossary:symbol]]// that is //[[CL:Glossary:apparently uninterned]]// is printed
preceded by ``\f{\#:}'' 
\issue{PRINT-READABLY-BEHAVIOR:CLARIFY}


if \varref{*print-gensym*} is //[[CL:Glossary:true]]// and //[[CL:Glossary:printer escaping]]// is enabled;
if \varref{*print-gensym*} is //[[CL:Glossary:false]]// or //[[CL:Glossary:printer escaping]]// is disabled,

then the //[[CL:Glossary:symbol]]// is printed without a prefix,
as if it were in the //[[CL:Glossary:current package]]//.


Because the \f{\#:} syntax does not intern the
following symbol, it is necessary to use circular-list syntax
if \varref{*print-circle*} is //[[CL:Glossary:true]]// and
the same uninterned symbol appears several times in an expression
to be printed.  For example, the result of

\code
 (let ((x (make-symbol "FOO"))) (list x x))
\endcode
would be printed as \f{(\#:foo \#:foo)} if \varref{*print-circle*}
were //[[CL:Glossary:false]]//, but as \f{(\#1=\#:foo \#1\#)} if \varref{*print-circle*}
were //[[CL:Glossary:true]]//.

A summary of the preceding package prefix rules follows:

\beginlist
\itemitem{\f{foo:bar}}

\f{foo:bar} is printed when //[[CL:Glossary:symbol]]// \f{bar} 
is external in its //[[CL:Glossary:home package]]// \f{foo} 
and is not //[[CL:Glossary:accessible]]// in the //[[CL:Glossary:current package]]//.
         
\itemitem{\f{foo::bar}}

\f{foo::bar} is printed when \f{bar} is internal in its //[[CL:Glossary:home package]]//
\f{foo} and is not //[[CL:Glossary:accessible]]// in the //[[CL:Glossary:current package]]//.
         
\itemitem{\f{:bar}}

\f{:bar} is printed when the home package of \f{bar} is \thepackage{keyword}.
                
\itemitem{\tt \#:bar}  

\f{\#:bar} is printed when \f{bar} is //[[CL:Glossary:apparently uninterned]]//,
even in the pathological case that \f{bar} 
has no //[[CL:Glossary:home package]]// but is nevertheless somehow //[[CL:Glossary:accessible]]// 
in the //[[CL:Glossary:current package]]//.
\endlist






\endsubsubsubsection%{Package Prefixes for Symbols}

\beginsubsubsubsection{Effect of Readtable Case on the Lisp Printer}
\DefineSection{ReadtableCasePrintEffect}

\issue{PRINT-CASE-BEHAVIOR:CLARIFY}

When 

//[[CL:Glossary:printer escaping]]// is disabled,
or the characters under consideration are not already 
quoted specifically by //[[CL:Glossary:single escape]]// or //[[CL:Glossary:multiple escape]]//
syntax,

the //[[CL:Glossary:readtable case]]// of the //[[CL:Glossary:current readtable]]// 
affects the way the //[[CL:Glossary:Lisp printer]]// writes //[[CL:Glossary:symbols]]//
in the following ways:
 
\beginlist
\itemitem{**'':upcase''**}

 When the //[[CL:Glossary:readtable case]]// is **'':upcase''**,
 //[[CL:Glossary:uppercase]]// //[[CL:Glossary:characters]]//
 are printed in the case specified by \varref{*print-case*}, and
 //[[CL:Glossary:lowercase]]// //[[CL:Glossary:characters]]// are printed in their own case.
 
\itemitem{**'':downcase''**}

 When the //[[CL:Glossary:readtable case]]// is **'':downcase''**,
 //[[CL:Glossary:uppercase]]// //[[CL:Glossary:characters]]// are printed in their own case, and
 //[[CL:Glossary:lowercase]]// //[[CL:Glossary:characters]]//
 are printed in the case specified by \varref{*print-case*}.
 
\itemitem{**'':preserve''**}

 When the //[[CL:Glossary:readtable case]]// is **'':preserve''**,
 all //[[CL:Glossary:alphabetic]]// //[[CL:Glossary:characters]]// are printed in their own case.
 
\itemitem{**'':invert''**}

 When the //[[CL:Glossary:readtable case]]// is **'':invert''**,
 the case of all //[[CL:Glossary:alphabetic]]// //[[CL:Glossary:characters]]// 
 in single case symbol names is inverted.
 Mixed-case symbol names are printed as is.
\endlist 

The rules for escaping //[[CL:Glossary:alphabetic]]// //[[CL:Glossary:characters]]// in symbol names are affected by
the **[[CL:Functions:readtable-case]]** 
\issue{PRINT-READABLY-BEHAVIOR:CLARIFY}

if //[[CL:Glossary:printer escaping]]// is enabled.

//[[CL:Glossary:Alphabetic]]// //[[CL:Glossary:characters]]// are escaped as follows:                
\beginlist
\itemitem{**'':upcase''**}

When the //[[CL:Glossary:readtable case]]// is **'':upcase''**,
all //[[CL:Glossary:lowercase]]// //[[CL:Glossary:characters]]// must be escaped.

\itemitem{**'':downcase''**}

When the //[[CL:Glossary:readtable case]]// is **'':downcase''**,
all //[[CL:Glossary:uppercase]]// //[[CL:Glossary:characters]]// must be escaped.

\itemitem{**'':preserve''**}

When the //[[CL:Glossary:readtable case]]// is **'':preserve''**, 
no //[[CL:Glossary:alphabetic]]// //[[CL:Glossary:characters]]// need be escaped.

\itemitem{**'':invert''**}

When the //[[CL:Glossary:readtable case]]// is **'':invert''**,
no //[[CL:Glossary:alphabetic]]// //[[CL:Glossary:characters]]// need be escaped.

\endlist    

\beginsubsubsubsubsection{Examples of Effect of Readtable Case on the Lisp Printer}
\DefineSection{ReadtableCasePrintExamples}

\code
 (defun test-readtable-case-printing ()
   (let ((*readtable* (copy-readtable nil))
         (*print-case* *print-case*))
     (format t "READTABLE-CASE *PRINT-CASE*  Symbol-name  Output~
              ~%--------------------------------------------------~
              ~%")
     (dolist (readtable-case '(:upcase :downcase :preserve :invert))
       (setf (readtable-case *readtable*) readtable-case)
       (dolist (print-case '(:upcase :downcase :capitalize))
         (dolist (symbol '(|ZEBRA| |Zebra| |zebra|))
           (setq *print-case* print-case)
           (format t "~&:~A~15T:~A~29T~A~42T~A"
                   (string-upcase readtable-case)
                   (string-upcase print-case)
                   (symbol-name symbol)
                   (prin1-to-string symbol)))))))
\endcode
  The output from \f{(test-readtable-case-printing)} should be as follows:

\code
    READTABLE-CASE *PRINT-CASE*  Symbol-name  Output
    --------------------------------------------------
    :UPCASE        :UPCASE       ZEBRA        ZEBRA
    :UPCASE        :UPCASE       Zebra        |Zebra|
    :UPCASE        :UPCASE       zebra        |zebra|
    :UPCASE        :DOWNCASE     ZEBRA        zebra
    :UPCASE        :DOWNCASE     Zebra        |Zebra|
    :UPCASE        :DOWNCASE     zebra        |zebra|
    :UPCASE        :CAPITALIZE   ZEBRA        Zebra
    :UPCASE        :CAPITALIZE   Zebra        |Zebra|
    :UPCASE        :CAPITALIZE   zebra        |zebra|
    :DOWNCASE      :UPCASE       ZEBRA        |ZEBRA|
    :DOWNCASE      :UPCASE       Zebra        |Zebra|
    :DOWNCASE      :UPCASE       zebra        ZEBRA
    :DOWNCASE      :DOWNCASE     ZEBRA        |ZEBRA|
    :DOWNCASE      :DOWNCASE     Zebra        |Zebra|
    :DOWNCASE      :DOWNCASE     zebra        zebra
    :DOWNCASE      :CAPITALIZE   ZEBRA        |ZEBRA|
    :DOWNCASE      :CAPITALIZE   Zebra        |Zebra|
    :DOWNCASE      :CAPITALIZE   zebra        Zebra
    :PRESERVE      :UPCASE       ZEBRA        ZEBRA
    :PRESERVE      :UPCASE       Zebra        Zebra
    :PRESERVE      :UPCASE       zebra        zebra
    :PRESERVE      :DOWNCASE     ZEBRA        ZEBRA
    :PRESERVE      :DOWNCASE     Zebra        Zebra
    :PRESERVE      :DOWNCASE     zebra        zebra
    :PRESERVE      :CAPITALIZE   ZEBRA        ZEBRA
    :PRESERVE      :CAPITALIZE   Zebra        Zebra
    :PRESERVE      :CAPITALIZE   zebra        zebra
    :INVERT        :UPCASE       ZEBRA        zebra
    :INVERT        :UPCASE       Zebra        Zebra
    :INVERT        :UPCASE       zebra        ZEBRA
    :INVERT        :DOWNCASE     ZEBRA        zebra
    :INVERT        :DOWNCASE     Zebra        Zebra
    :INVERT        :DOWNCASE     zebra        ZEBRA
    :INVERT        :CAPITALIZE   ZEBRA        zebra
    :INVERT        :CAPITALIZE   Zebra        Zebra
    :INVERT        :CAPITALIZE   zebra        ZEBRA
\endcode

\endsubsubsubsubsection%{Examples of Effect of Readtable Case on the Lisp Printer}

\endsubsubsubsection%{Effect of Readtable Case on the Lisp Printer}

\endsubsubsection%{Printing Symbols}
\beginsubsubsection{Printing Strings}
\DefineSection{PrintingStrings}


The characters of the //[[CL:Glossary:string]]// are output in order.
\issue{PRINT-READABLY-BEHAVIOR:CLARIFY}

If //[[CL:Glossary:printer escaping]]// is enabled,

a //[[CL:Glossary:double-quote]]// is output before and after, and all
//[[CL:Glossary:double-quotes]]// and //[[CL:Glossary:single escapes]]// are preceded by //[[CL:Glossary:backslash]]//.
The printing of //[[CL:Glossary:strings]]// is not affected by \varref{*print-array*}.
Only the //[[CL:Glossary:active]]// //[[CL:Glossary:elements]]// of the //[[CL:Glossary:string]]// are printed.

For information on how the //[[CL:Glossary:Lisp reader]]// parses //[[CL:Glossary:strings]]//,
\seesection\Doublequote.

\endsubsubsection%{Printing Strings}
\beginsubsubsection{Printing Lists and Conses}
\DefineSection{PrintingListsAndConses}


Wherever possible, list notation is preferred over dot notation.  
Therefore the following algorithm is used to print a //[[CL:Glossary:cons]]// $x$:

\goodbreak
\beginlist
\item{1.} A //[[CL:Glossary:left-parenthesis]]// is printed.

\medbreak
\item{2.} The //[[CL:Glossary:car]]// of $x$ is printed. 

\medbreak
\item{3.} If the //[[CL:Glossary:cdr]]// of $x$ is itself a //[[CL:Glossary:cons]]//,
          it is made to be the current //[[CL:Glossary:cons]]// 
	  (\ie $x$ becomes that //[[CL:Glossary:cons]]//), 
\issue{PRINT-WHITESPACE:JUST-ONE-SPACE}
	  a //[[CL:Glossary:space]]//


	  is printed,
          and step 2 is re-entered.

\medbreak
\item{4.} If the //[[CL:Glossary:cdr]]// of $x$ is not //[[CL:Glossary:null]]//, 
\issue{PRINT-WHITESPACE:JUST-ONE-SPACE}
	  a //[[CL:Glossary:space]]//,


          a //[[CL:Glossary:dot]]//,
\issue{PRINT-WHITESPACE:JUST-ONE-SPACE}
	  a //[[CL:Glossary:space]]//,


          and the //[[CL:Glossary:cdr]]// of $x$ are printed.

\medbreak
\item{5.} A //[[CL:Glossary:right-parenthesis]]// is printed.
\endlist

\issue{PRINT-WHITESPACE:JUST-ONE-SPACE}
Actually, the above algorithm is only used when \varref{*print-pretty*}
is //[[CL:Glossary:false]]//.  When \varref{*print-pretty*} is //[[CL:Glossary:true]]// (or 
when **[[CL:Functions:pprint]]** is used),
additional //[[CL:Glossary:whitespace]]//\meaning{1} 
may replace the use of a single //[[CL:Glossary:space]]//,
and a more elaborate algorithm with similar goals but more presentational 
flexibility is used; \seesection\PrinterDispatch.




Although the two expressions below are equivalent,
and the reader accepts
either one and 


produces
the same //[[CL:Glossary:cons]]//, the printer
always prints such a //[[CL:Glossary:cons]]// in the second form.

\code
 (a . (b . ((c . (d . nil)) . (e . nil))))
 (a b (c d) e)
\endcode
The printing of //[[CL:Glossary:conses]]// is affected by \varref{*print-level*},
\varref{*print-length*}, and \varref{*print-circle*}.


\goodbreak         
Following are examples of printed representations of //[[CL:Glossary:lists]]//:

\code
 (a . b)     ;A dotted pair of a and b
 (a.b)       ;A list of one element, the symbol named a.b
 (a. b)      ;A list of two elements a. and b
 (a .b)      ;A list of two elements a and .b
 (a b . c)   ;A dotted list of a and b with c at the end; two conses
 .iot        ;The symbol whose name is .iot
 (. b)       ;Invalid -- an error is signaled if an attempt is made to read 
             ;this syntax.
 (a .)       ;Invalid -- an error is signaled.
 (a .. b)    ;Invalid -- an error is signaled.
 (a . . b)   ;Invalid -- an error is signaled.
 (a b c ...) ;Invalid -- an error is signaled.
 (a \\. b)    ;A list of three elements a, ., and b
 (a |.| b)   ;A list of three elements a, ., and b
 (a \\... b)  ;A list of three elements a, ..., and b
 (a |...| b) ;A list of three elements a, ..., and b
\endcode

For information on how the //[[CL:Glossary:Lisp reader]]// parses //[[CL:Glossary:lists]]// and //[[CL:Glossary:conses]]//,
\seesection\LeftParen. 
     
\endsubsubsection%{Printing Lists and Conses}
\beginsubsubsection{Printing Bit Vectors}
\DefineSection{PrintingBitVectors}


A //[[CL:Glossary:bit vector]]// is printed as \f{\#*} followed by the bits of the //[[CL:Glossary:bit vector]]//
in order.  If \varref{*print-array*} is //[[CL:Glossary:false]]//, then the //[[CL:Glossary:bit vector]]// is
printed in a format (using \f{\#<}) that is concise but not readable.
Only the //[[CL:Glossary:active]]// //[[CL:Glossary:elements]]// of the //[[CL:Glossary:bit vector]]// are printed.

\reviewer{Barrett: Need to provide for \f{\#5*0} as an alternate 
  	  notation for \f{\#*00000}.}%!!!


For information on //[[CL:Glossary:Lisp reader]]// parsing of //[[CL:Glossary:bit vectors]]//,
\seesection\SharpsignStar.

\endsubsubsection%{Printing Bit Vectors}
\beginsubsubsection{Printing Other Vectors}
\DefineSection{PrintingOtherVectors}


\issue{PRINT-READABLY-BEHAVIOR:CLARIFY}

If \varref{*print-array*} is //[[CL:Glossary:true]]// 
and \varref{*print-readably*} is //[[CL:Glossary:false]]//,
any

//[[CL:Glossary:vector]]// 
other than a //[[CL:Glossary:string]]// or //[[CL:Glossary:bit vector]]// is printed using
general-vector syntax; this means that information
about specialized vector representations does not appear.
The printed representation of a zero-length //[[CL:Glossary:vector]]// is \f{\#()}.
The printed representation of a non-zero-length //[[CL:Glossary:vector]]// begins with \f{\#(}.
Following that, the first element of the //[[CL:Glossary:vector]]// is printed.  
\issue{PRINTER-WHITESPACE:JUST-ONE-SPACE}
If there are any other elements, they are printed in turn, with 
each such additional element preceded by
a //[[CL:Glossary:space]]// if \varref{*print-pretty*} is //[[CL:Glossary:false]]//,
or //[[CL:Glossary:whitespace]]//\meaning{1} if \varref{*print-pretty*} is //[[CL:Glossary:true]]//.

A //[[CL:Glossary:right-parenthesis]]// after the last element
terminates the printed representation of the //[[CL:Glossary:vector]]//. 
The printing of //[[CL:Glossary:vectors]]// 
is affected by \varref{*print-level*} and \varref{*print-length*}.
If the //[[CL:Glossary:vector]]// has a //[[CL:Glossary:fill pointer]]//, 
then only those elements below
the //[[CL:Glossary:fill pointer]]// are printed.


\issue{PRINT-READABLY-BEHAVIOR:CLARIFY}

If both \varref{*print-array*} and \varref{*print-readably*} are //[[CL:Glossary:false]]//,

the //[[CL:Glossary:vector]]// is not printed as described above,
but in a format (using \f{\#<}) that is concise but not readable.

\issue{PRINT-READABLY-BEHAVIOR:CLARIFY}
If \varref{*print-readably*} is //[[CL:Glossary:true]]//,
the //[[CL:Glossary:vector]]// prints in an //[[CL:Glossary:implementation-defined]]// manner;
\seevar{*print-readably*}.


For information on how the //[[CL:Glossary:Lisp reader]]// parses these ``other //[[CL:Glossary:vectors]]//,''
\seesection\SharpsignLeftParen.

\endsubsubsection%{Printing Other Vectors}
\beginsubsubsection{Printing Other Arrays}
\DefineSection{PrintingOtherArrays}


\issue{PRINT-READABLY-BEHAVIOR:CLARIFY}

If  \varref{*print-array*} is //[[CL:Glossary:true]]// 
and \varref{*print-readably*} is //[[CL:Glossary:false]]//,
any

//[[CL:Glossary:array]]// other than a //[[CL:Glossary:vector]]// is printed
using \f{\#}\f{n}\f{A} format.
Let \f{n} be the //[[CL:Glossary:rank]]// of the //[[CL:Glossary:array]]//.
Then \f{\#} is printed, then \f{n} as a decimal integer,
then \f{A}, then \f{n} open parentheses.  
Next the //[[CL:Glossary:elements]]// are scanned in row-major order,

using **[[CL:Functions:write]]** on each //[[CL:Glossary:element]]//, 
and separating //[[CL:Glossary:elements]]// from each other with //[[CL:Glossary:whitespace]]//\meaning{1}.


The array's dimensions are numbered 0 to \f{n}-1 from left to right,
and are enumerated with the rightmost index changing fastest.


Every time the index for dimension \f{j} is incremented,
the following actions are taken:


\beginlist
\itemitem{\bull}
If \f{j} < \f{n}-1, then a close parenthesis is printed.


\itemitem{\bull}
If incrementing the index for dimension \f{j} caused it to equal
dimension \f{j}, that index is reset to zero and the
index for dimension \f{j}-1 is incremented (thereby performing these three steps recursively),
unless \f{j}=0, in which case the entire algorithm is terminated.
If incrementing the index for dimension \f{j} did not cause it to
equal dimension \f{j}, then a space is printed.


\itemitem{\bull}
If \f{j} < \f{n}-1, then an open parenthesis is printed.
\endlist

This causes the contents to be printed in a format suitable for
**'':initial-contents''** to **[[CL:Functions:make-array]]**.
The lists effectively printed by this procedure are subject to
truncation by \varref{*print-level*} and \varref{*print-length*}.


If the //[[CL:Glossary:array]]// 
is of a specialized //[[CL:Glossary:type]]//, containing bits or characters,
then the innermost lists generated by the algorithm given above can instead
be printed using bit-vector or string syntax, provided that these innermost
lists would not be subject to truncation by \varref{*print-length*}.  


















\issue{PRINT-READABLY-BEHAVIOR:CLARIFY}

If both \varref{*print-array*} and \varref{*print-readably*} are //[[CL:Glossary:false]]//,

then the //[[CL:Glossary:array]]// is printed
in a format (using \f{\#<}) that is concise but not readable.

\issue{PRINT-READABLY-BEHAVIOR:CLARIFY}
If \varref{*print-readably*} is //[[CL:Glossary:true]]//,
the //[[CL:Glossary:array]]// prints in an //[[CL:Glossary:implementation-defined]]// manner; 
\seevar{*print-readably*}.


In particular,
this may be important for arrays having some dimension \f{0}.

For information on how the //[[CL:Glossary:Lisp reader]]// parses these ``other //[[CL:Glossary:arrays]]//,''
see \secref\SharpsignA.

\beginsubsubsection{Examples of Printing Arrays}

\code
 (let ((a (make-array '(3 3)))
       (*print-pretty* t)
       (*print-array* t))
   (dotimes (i 3) (dotimes (j 3) (setf (aref a i j) (format nil "<~D,~D>" i j))))
   (print a)
   (print (make-array 9 :displaced-to a)))
\OUT #2A(("<0,0>" "<0,1>" "<0,2>") 
\OUT     ("<1,0>" "<1,1>" "<1,2>") 
\OUT     ("<2,0>" "<2,1>" "<2,2>")) 
\OUT #("<0,0>" "<0,1>" "<0,2>" "<1,0>" "<1,1>" "<1,2>" "<2,0>" "<2,1>" "<2,2>") 
\EV #<ARRAY 9 indirect 36363476>
\endcode

\endsubsubsection%{Examples of Printing Arrays}

\endsubsubsection%{Printing Other Arrays}
\beginsubsubsection{Printing Random States}
\DefineSection{PrintingRandomStates}


                                                                       
A specific syntax for printing //[[CL:Glossary:objects]]// \oftype{random-state} is
not specified. However, every //[[CL:Glossary:implementation]]//
must arrange to print a //[[CL:Glossary:random state]]// //[[CL:Glossary:object]]// in such a way that,
within the same implementation, **[[CL:Functions:read]]**
can construct from the printed representation a copy of the 
//[[CL:Glossary:random state]]//
object as if the copy had been made by **[[CL:Functions:make-random-state]]**.

If the type //[[CL:Glossary:random state]]// is effectively implemented 
by using the machinery for \macref{defstruct},
the usual structure syntax can then be used for printing 
//[[CL:Glossary:random state]]//
objects; one might look something like





\code
 #S(RANDOM-STATE :DATA #(14 49 98436589 786345 8734658324 ... ))
\endcode
where the components are //[[CL:Glossary:implementation-dependent]]//.

\endsubsubsection%{Printing Random States}
\beginsubsubsection{Printing Pathnames}
\DefineSection{PrintingPathnames}


\issue{PATHNAME-PRINT-READ:SHARPSIGN-P}
 
\issue{PRINT-READABLY-BEHAVIOR:CLARIFY}

When //[[CL:Glossary:printer escaping]]// is enabled,

the syntax \f{\#P"..."} is how a
//[[CL:Glossary:pathname]]// is printed by **[[CL:Functions:write]]** and the other functions herein described.

The \f{"..."} is the namestring representation of the pathname.
 









\issue{PRINT-READABLY-BEHAVIOR:CLARIFY}

When //[[CL:Glossary:printer escaping]]// is disabled,

**[[CL:Functions:write]]** writes a //[[CL:Glossary:pathname]]// \i{P}
by writing \f{(namestring \i{P})} instead.















For information on how the //[[CL:Glossary:Lisp reader]]// parses //[[CL:Glossary:pathnames]]//,
see \secref\SharpsignP.



\endsubsubsection%{Printing Pathnames}
\beginsubsubsection{Printing Structures}
\DefineSection{PrintingStructures}

\issue{DEFSTRUCT-PRINT-FUNCTION-AGAIN:X3J13-MAR-93}






By default, a //[[CL:Glossary:structure]]// of type $S$ is printed using \f{\#S} syntax.
This behavior can be customized by specifying a **'':print-function''** 
or **'':print-object''** option to the \macref{defstruct} //[[CL:Glossary:form]]// that defines $S$,
or by writing a **[[CL:Functions:print-object]]** //[[CL:Glossary:method]]// 
that is //[[CL:Glossary:specialized]]// for //[[CL:Glossary:objects]]// of type $S$.


\issue{STRUCTURE-READ-PRINT-SYNTAX:KEYWORDS}

Different structures might print out in different ways;
the default notation for structures is:

\code
 #S(//structure-name// \star{\curly{//slot-key// //slot-value//}})
\endcode
where \f{\#S} indicates structure syntax,
//structure-name// is a //[[CL:Glossary:structure name]]//,
each //slot-key// is an initialization argument //[[CL:Glossary:name]]//
for a //[[CL:Glossary:slot]]// in the //[[CL:Glossary:structure]]//,
and each corresponding //slot-value// is a representation
of the //[[CL:Glossary:object]]// in that //[[CL:Glossary:slot]]//.



For information on how the //[[CL:Glossary:Lisp reader]]// parses //[[CL:Glossary:structures]]//,
see \secref\SharpsignS.

\endsubsubsection%{Printing Structures}
\beginsubsubsection{Printing Other Objects}
\DefineSection{PrintingOtherObjects}



Other //[[CL:Glossary:objects]]// are printed in an //[[CL:Glossary:implementation-dependent]]// manner.
It is not required that an //[[CL:Glossary:implementation]]// print those //[[CL:Glossary:objects]]//
//[[CL:Glossary:readably]]//.

For example, //[[CL:Glossary:hash tables]]//, 
	     //[[CL:Glossary:readtables]]//,
             //[[CL:Glossary:packages]]//,
             //[[CL:Glossary:streams]]//,
         and //[[CL:Glossary:functions]]//
might not print //[[CL:Glossary:readably]]//.

A common notation to use in this circumstance is \f{\#<...>}.
Since \f{\#<} is not readable by the //[[CL:Glossary:Lisp reader]]//,
the precise format of the text which follows is not important,
but a common format to use is that provided by \themacro{print-unreadable-object}.

For information on how the //[[CL:Glossary:Lisp reader]]// treats this notation,
\seesection\SharpsignLeftAngle.
For information on how to notate //[[CL:Glossary:objects]]// that cannot be printed //[[CL:Glossary:readably]]//,
\seesection\SharpsignDot.

\endsubsubsection%{Printing Other Objects}

\endsubsection%{Default Print-Object Methods}




\beginsubSection{Examples of Printer Behavior}

\code
 (let ((*print-escape* t)) (fresh-line) (write #\\a))
\OUT #\\a
\EV #\\a
 (let ((*print-escape* nil) (*print-readably* nil))
   (fresh-line)
   (write #\\a))
\OUT a
\EV #\\a
 (progn (fresh-line) (prin1 #\\a))
\OUT #\\a
\EV #\\a
 (progn (fresh-line) (print #\\a))
\OUT 
\OUT #\\a
\EV #\\a
 (progn (fresh-line) (princ #\\a))
\OUT a
\EV #\\a
\medbreak
 (dolist (val '(t nil))
   (let ((*print-escape* val) (*print-readably* val))
     (print '#\\a) 
     (prin1 #\\a) (write-char #\\Space)
     (princ #\\a) (write-char #\\Space)
     (write #\\a)))
\OUT #\\a #\\a a #\\a
\OUT #\\a #\\a a a
\EV NIL
\medbreak
 (progn (fresh-line) (write '(let ((a 1) (b 2)) (+ a b))))
\OUT (LET ((A 1) (B 2)) (+ A B))
\EV (LET ((A 1) (B 2)) (+ A B))
\medbreak
 (progn (fresh-line) (pprint '(let ((a 1) (b 2)) (+ a b))))
\OUT (LET ((A 1)
\OUT       (B 2))               
\OUT   (+ A B))
\EV (LET ((A 1) (B 2)) (+ A B))
\medbreak
 (progn (fresh-line) 
        (write '(let ((a 1) (b 2)) (+ a b)) :pretty t))
\OUT (LET ((A 1)
\OUT       (B 2))
\OUT   (+ A B))                 
\EV (LET ((A 1) (B 2)) (+ A B))
\medbreak
 (with-output-to-string (s)  
    (write 'write :stream s)
    (prin1 'prin1 s))
\EV "WRITEPRIN1"
\endcode

\endsubSection%{Examples of Printer Behavior}

