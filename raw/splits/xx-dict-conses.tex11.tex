====== Function SUBLIS, NSUBLIS ======

====Syntax====

**sublis {alist tree** //\key} key test test-not// → //new-tree// **nsublis {alist tree** //\key} key test test-not// → //new-tree//

====Arguments and Values====

//alist// - an //[[CL:Glossary:association list]]//.

//tree// - a //[[CL:Glossary:tree]]//.

//test// - a //[[CL:Glossary:designator]]// for a //[[CL:Glossary:function]]// of two //[[CL:Glossary:arguments]]// that returns a //[[CL:Glossary:generalized boolean]]//.

//test-not// - a //[[CL:Glossary:designator]]// for a //[[CL:Glossary:function]]// of two //[[CL:Glossary:arguments]]// that returns a //[[CL:Glossary:generalized boolean]]//.

//key// - a //[[CL:Glossary:designator]]// for a //[[CL:Glossary:function]]// of one argument, or **[[CL:Constant Variables:nil]]**.

//new-tree// - a //[[CL:Glossary:tree]]//.

====Description====

**[[CL:Functions:sublis]]** makes substitutions for //[[CL:Glossary:object|objects]]// in //tree// (a structure of //[[CL:Glossary:conses]]//).

**[[CL:Functions:nsublis]]** is like **[[CL:Functions:sublis]]** but destructively modifies the relevant parts of the //tree//.

**[[CL:Functions:sublis]]** looks at all subtrees and leaves of //tree//; if a subtree or leaf appears as a key in //alist// (that is, the key and the subtree or leaf //[[CL:Glossary:satisfy the test]]//), it is replaced by the //[[CL:Glossary:object]]// with which that key is associated. This operation is non-destructive. In effect, **[[CL:Functions:sublis]]** can perform several **[[CL:Functions:subst]]** operations simultaneously.

If **[[CL:Functions:sublis]]** succeeds, a new copy of //tree// is returned in which each occurrence of such a subtree or leaf is replaced by the //[[CL:Glossary:object]]// with which it is associated. If no changes are made, the original tree is returned. The original //tree// is left unchanged, but the result tree may share cells with it.

**[[CL:Functions:nsublis]]** is permitted to modify //tree// but otherwise returns the same values as **[[CL:Functions:sublis]]**.

====Examples====

<blockquote> (sublis '((x . 100) (z . zprime)) '(plus x (minus g z x p) 4 . x)) → (PLUS 100 (MINUS G ZPRIME 100 P) 4 . 100) (sublis '(((+ x y) . (- x y)) ((- x y) . (+ x y))) '(* (/ (+ x y) (+ x p)) (- x y)) :test #'equal) → (* (/ (- X Y) (+ X P)) (+ X Y)) ([[CL:Macros:defparameter]] tree1 '(1 (1 2) ((1 2 3)) (((1 2 3 4))))) → (1 (1 2) ((1 2 3)) (((1 2 3 4)))) (sublis '((3 . "three")) tree1) → (1 (1 2) ((1 2 "three")) (((1 2 "three" 4)))) (sublis '((t . "string")) (sublis '((1 . "") (4 . 44)) tree1) :key #'stringp) → ("string" ("string" 2) (("string" 2 3)) ((("string" 2 3 44)))) tree1 → (1 (1 2) ((1 2 3)) (((1 2 3 4)))) ([[CL:Macros:defparameter]] tree2 '("one" ("one" "two") (("one" "Two" "three")))) → ("one" ("one" "two") (("one" "Two" "three"))) (sublis '(("two" . 2)) tree2) → ("one" ("one" "two") (("one" "Two" "three"))) tree2 → ("one" ("one" "two") (("one" "Two" "three"))) (sublis '(("two" . 2)) tree2 :test 'equal) → ("one" ("one" 2) (("one" "Two" "three")))

(nsublis '((t . 'temp)) tree1 :key #'(lambda (x) (or (atom x) (< (list-length x) 3)))) → ((QUOTE TEMP) (QUOTE TEMP) QUOTE TEMP) </blockquote>

====Side Effects====

**[[CL:Functions:nsublis]]** modifies //tree//.

====Affected By====

None.

====Exceptional Situations====

None.

====See Also====

**[[CL:Functions:subst]]**,

{\secref\ConstantModification},

{\secref\TraversalRules}

====Notes====

The **'':test-not''** parameter is deprecated.

Because the side-effecting variants (//e.g.// **[[CL:Functions:nsublis]]**) potentially change the path that is being traversed, their effects in the presence of shared or circular structure structure may vary in surprising ways when compared to their non-side-effecting alternatives. To see this, consider the following side-effect behavior, which might be exhibited by some implementations:

<blockquote> (defun test-it (fn) (let* ((shared-piece (list 'a 'b)) (data (list shared-piece shared-piece))) (funcall fn '((a . b) (b . a)) data))) (test-it #'sublis) → ((B A) (B A)) (test-it #'nsublis) → ((A B) (A B)) </blockquote>

\issue{TEST-NOT-IF-NOT:FLUSH-ALL} \issue{DOTTED-LIST-ARGUMENTS:CLARIFY} \issue{DOTTED-LIST-ARGUMENTS:CLARIFY} \issue{CONSTANT-MODIFICATION:DISALLOW} \issue{MAPPING-DESTRUCTIVE-INTERACTION:EXPLICITLY-VAGUE}
