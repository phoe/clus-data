====== Condition Type STYLE-WARNING ======

====Class Precedence List==== **[[CL:Types:style-warning]]**, **[[CL:Types:warning]]**, **[[CL:Types:condition]]**, **[[CL:Types:t]]**

====Description====

The type **[[CL:Types:style-warning]]** includes those //[[CL:Glossary:conditions]]// that represent //[[CL:Glossary:situations]]// involving //[[CL:Glossary:code]]// that is //[[CL:Glossary:conforming code]]// but that is nevertheless considered to be faulty or substandard.

====See Also====

**[[CL:Functions:muffle-warning]]**

====Notes====

An //[[CL:Glossary:implementation]]// might signal such a //[[CL:Glossary:condition]]// if it encounters //[[CL:Glossary:code]]// that uses deprecated features or that appears unaesthetic or inefficient.

An `unused variable' warning must be of type **[[CL:Types:style-warning]]**.

In general, the question of whether //[[CL:Glossary:code]]// is faulty or substandard is a subjective decision to be made by the facility processing that //[[CL:Glossary:code]]//. The intent is that whenever such a facility wishes to complain about //[[CL:Glossary:code]]// on such subjective grounds, it should use this //[[CL:Glossary:condition]]// //[[CL:Glossary:type]]// so that any clients who wish to redirect or muffle superfluous warnings can do so without risking that they will be redirecting or muffling other, more serious warnings.

