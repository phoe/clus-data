====== Function VECTOR-PUSH, VECTOR-PUSH-EXTEND ======

====Syntax====

**vector-push** //new-element vector// → //new-index-p//

**vector-push-extend** //new-element vector ''&optional'' extension// → //new-index//

====Arguments and Values====

//new-element// - an //[[CL:Glossary:object]]//.

//vector// - a //[[CL:Glossary:vector]]// with a //[[CL:Glossary:fill pointer]]//.

//extension// - a positive //[[CL:Glossary:integer]]//. The default is //[[CL:Glossary:implementation-dependent]]//.

//new-index-p// - a //[[CL:Glossary:valid array index]]// for //vector//, or **[[CL:Constant Variables:nil]]**.

//new-index// - a //[[CL:Glossary:valid array index]]// for //vector//.

====Description====

**[[CL:Functions:vector-push]]** and **[[CL:Functions:vector-push-extend]]** store //new-element// in //vector//. **[[CL:Functions:vector-push]]** attempts to store //new-element// in the element of //vector// designated by the //[[CL:Glossary:fill pointer]]//, and to increase the //[[CL:Glossary:fill pointer]]// by one. If the ''(>= (fill-pointer //vector//) (array-dimension //vector// 0))'', neither //vector// nor its //[[CL:Glossary:fill pointer]]// are affected. Otherwise, the store and increment take place and **[[CL:Functions:vector-push]]** returns the former value of the //[[CL:Glossary:fill pointer]]// which is one less than the one it leaves in //vector//.

**[[CL:Functions:vector-push-extend]]** is just like **[[CL:Functions:vector-push]]** except that if the //[[CL:Glossary:fill pointer]]// gets too large, //vector// is extended using **[[CL:Functions:adjust-array]]** so that it can contain more elements. //Extension// is the minimum number of elements to be added to //vector// if it must be extended.

**[[CL:Functions:vector-push]]** and **[[CL:Functions:vector-push-extend]]** return the index of //new-element// in //vector//. If ''(>= (fill-pointer //vector//) (array-dimension //vector// 0))'', **[[CL:Functions:vector-push]]** returns **[[CL:Constant Variables:nil]]**.

====Examples====

<blockquote> (vector-push ([[CL:Macros:defparameter]] fable (list 'fable)) ([[CL:Macros:defparameter]] fa (make-array 8 :fill-pointer 2 :initial-element 'first-one))) → 2 (fill-pointer fa) → 3 (eq (aref fa 2) fable) → //[[CL:Glossary:true]]// (vector-push-extend #\X ([[CL:Macros:defparameter]] aa (make-array 5 :element-type 'character :adjustable t :fill-pointer 3))) → 3 (fill-pointer aa) → 4 (vector-push-extend #\Y aa 4) → 4 (array-total-size aa) → at least 5 (vector-push-extend #\Z aa 4) → 5 (array-total-size aa) → 9 ;(or more) </blockquote>

====Affected By==== The value of the //[[CL:Glossary:fill pointer]]//.

How //vector// was created.

====Exceptional Situations====

An error of type **[[CL:Types:error]]** is signaled by **[[CL:Functions:vector-push-extend]]** if it tries to extend //vector// and //vector// is not //[[CL:Glossary:actually adjustable]]//.

An error of type **[[CL:Types:error]]** is signaled if //vector// does not have a //[[CL:Glossary:fill pointer]]//.

====See Also====

**[[CL:Functions:adjustable-array-p]]**, **[[CL:Functions:fill-pointer]]**, **[[CL:Functions:vector-pop]]**

====Notes====

None.

