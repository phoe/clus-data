====== Variable *READTABLE* ======

====Value Type====

a //[[CL:Glossary:readtable]]//.

====Initial Value====

A //[[CL:Glossary:readtable]]// that conforms to the description of \clisp\ syntax in \chapref\Syntax.

====Description====

\Thevalueof{*readtable*} is called the //[[CL:Glossary:current readtable]]//. It controls the parsing behavior of the //[[CL:Glossary:Lisp reader]]//, and can also influence the //[[CL:Glossary:Lisp printer]]// (//e.g.// see the //[[CL:Glossary:function]]// **[[CL:Functions:readtable-case]]**).

====Examples====

<blockquote> (readtablep *readtable*) → //[[CL:Glossary:true]]// ([[CL:Macros:defparameter]] zvar 123) → 123 (set-syntax-from-char #\\z #\\' ([[CL:Macros:defparameter]] table2 (copy-readtable))) → T zvar → 123 ([[CL:Macros:defparameter]] *readtable* table2) → #<READTABLE> zvar → VAR ([[CL:Macros:defparameter]] *readtable* (copy-readtable nil)) → #<READTABLE> zvar → 123 </blockquote>

====Affected By====

**[[CL:Functions:compile-file]]**, **[[CL:Functions:load]]**

====See Also====

**[[CL:Functions:compile-file]]**, **[[CL:Functions:load]]**, **[[CL:Functions:readtable]]**, {\secref\CurrentReadtable}

====Notes====

None.

