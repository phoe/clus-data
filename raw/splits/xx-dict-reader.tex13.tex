====== Variable *READ-BASE* ======

====Value Type====

a //[[CL:Glossary:radix]]//.

====Initial Value====

''10''.

====Description====

Controls the interpretation of tokens by **[[CL:Functions:read]]** as being //[[CL:Glossary:integers]]// or //[[CL:Glossary:ratios]]//.

\Thevalueof{*read-base*}, called the //[[CL:Glossary:current input base]]//, is the radix in which //[[CL:Glossary:integers]]// and //[[CL:Glossary:ratios]]// are to be read by the //[[CL:Glossary:Lisp reader]]//. The parsing of other numeric //[[CL:Glossary:types]]// (//e.g.// //[[CL:Glossary:floats]]//) is not affected by this option.

The effect of **[[CL:Variables:*read-base*]]** on the reading of any particular //[[CL:Glossary:rational]]// number can be locally overridden by explicit use of the ''#O'', ''#X'', ''#B'', or **[[CL:Functions:#''n''R]]** syntax or by a trailing decimal point.

====Examples====

<blockquote> (dotimes (i 6) (let ((*read-base* (+ 10. i))) (let ((object (read-from-string "(\\\\DAD DAD |BEE| BEE 123. 123)"))) (print (list *read-base* object)))))
▷ (10 (DAD DAD BEE BEE 123 123))
▷ (11 (DAD DAD BEE BEE 123 146))
▷ (12 (DAD DAD BEE BEE 123 171))
▷ (13 (DAD DAD BEE BEE 123 198))
▷ (14 (DAD 2701 BEE BEE 123 227))
▷ (15 (DAD 3088 BEE 2699 123 258)) → NIL </blockquote>

====Affected By====

None.

====See Also====

None.

====Notes====

Altering the input radix can be useful when reading data files in special formats.

