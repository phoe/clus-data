====== Declaration SPECIAL ======

====Syntax====

\f{(special \starparam{var})}

====Arguments====

//var// - a //[[CL:Glossary:symbol]]//.

====Valid Context====

//[[CL:Glossary:declaration]]// or //[[CL:Glossary:proclamation]]//

====Binding Types Affected====

//[[CL:Glossary:variable]]//

====Description====

Specifies that all of the //vars// named are dynamic. This specifier affects variable //[[CL:Glossary:bindings]]// and affects references. All variable //[[CL:Glossary:bindings]]// affected are made to be dynamic //[[CL:Glossary:bindings]]//, and affected variable references refer to the current dynamic //[[CL:Glossary:binding]]//.

For example:

<blockquote> (defun hack (thing *mod*) ;The binding of the parameter (declare (special *mod*)) ; *mod* is visible to hack1, (hack1 (car thing))) ; but not that of thing. (defun hack1 (arg) (declare (special *mod*)) ;Declare references to *mod* ;within hack1 to be special. (if (atom arg) *mod* (cons (hack1 (car arg)) (hack1 (cdr arg))))) </blockquote>

A \declref{special} declaration does not affect inner //[[CL:Glossary:bindings]]// of a //var//; the inner //[[CL:Glossary:bindings]]// implicitly shadow a \declref{special} declaration and must be explicitly re-declared to be \declref{special}. \declref{special} declarations never apply to function //[[CL:Glossary:bindings]]//.

\declref{special} declarations can be either //[[CL:Glossary:bound declarations]]//, affecting both a binding and references, or //[[CL:Glossary:free declarations]]//, affecting only references, depending on whether the declaration is attached to a variable binding.

When used in a //[[CL:Glossary:proclamation]]//, a \declref{special} //[[CL:Glossary:declaration specifier]]// applies to all //[[CL:Glossary:bindings]]// as well as to all references of the mentioned variables. For example, after

<blockquote> (declaim (special x)) </blockquote>

then in a function definition such as

<blockquote> (defun example (x) ...) </blockquote>

the parameter ''x'' is bound as a dynamic variable rather than as a lexical variable.

====Examples====

<blockquote> (defun declare-eg (y) ;this y is special (declare (special y)) (let ((y t)) ;this y is lexical (list y (locally (declare (special y)) y)))) ;this y refers to the ;special binding of y → DECLARE-EG (declare-eg nil) → (T NIL) </blockquote>

<blockquote> ([[CL:Macros:setf]] (symbol-value 'x) 6) (defun foo (x) ;a lexical binding of x (print x) (let ((x (1+ x))) ;a special binding of x (declare (special x)) ;and a lexical reference (bar)) (1+ x)) (defun bar () (print (locally (declare (special x)) x))) (foo 10)
▷ 10
▷ 11 → 11 </blockquote>

<blockquote> ([[CL:Macros:setf]] (symbol-value 'x) 6) (defun bar (x y) ;[1] 1st occurrence of x (let ((old-x x) ;[2] 2nd occurrence of x -- same as 1st occurrence (x y)) ;[3] 3rd occurrence of x (declare (special x)) (list old-x x))) (bar 'first 'second) → (FIRST SECOND) </blockquote>

<blockquote> (defun few (x &optional (y *foo*)) (declare (special *foo*)) ...) </blockquote> The reference to ''*foo*'' in the first line of this example is not \declref{special} even though there is a \declref{special} declaration in the second line.

<blockquote> (declaim (special prosp)) → //[[CL:Glossary:implementation-dependent]]// ([[CL:Macros:defparameter]] prosp 1 reg 1) → 1 (let ((prosp 2) (reg 2)) ;the binding of prosp is special (set 'prosp 3) (set 'reg 3) ;due to the preceding proclamation, (list prosp reg)) ;whereas the variable reg is lexical → (3 2) (list prosp reg) → (1 3)

(declaim (special x)) ;x is always special. (defun example (x y) (declare (special y)) (let ((y 3) (x (* x 2))) (print (+ y (locally (declare (special y)) y))) (let ((y 4)) (declare (special y)) (foo x)))) → EXAMPLE </blockquote> In the contorted code above, the outermost and innermost //[[CL:Glossary:bindings]]// of ''y'' are dynamic, but the middle binding is lexical. The two arguments to ''+'' are different, one being the value, which is ''3'', of the lexical variable ''y'', and the other being the value of the dynamic variable named ''y'' (a //[[CL:Glossary:binding]]// of which happens, coincidentally, to lexically surround it at an outer level). All the //[[CL:Glossary:bindings]]// of ''x'' and references to ''x'' are dynamic, however, because of the proclamation that ''x'' is always \declref{special}.

====See Also====

**[[CL:Macros:defparameter]]**, **[[CL:Macros:defvar]]**

