====== Function READ-DELIMITED-LIST ======

====Syntax====

\DefunWithValues read-delimited-list {char ''&optional'' input-stream recursive-p} {list}

====Arguments and Values====

//char// - a //[[CL:Glossary:character]]//.

//input-stream// - an //[[CL:Glossary:input]]// //[[CL:Glossary:stream designator]]//. The default is //[[CL:Glossary:standard input]]//.

//recursive-p// - a //[[CL:Glossary:generalized boolean]]//. The default is //[[CL:Glossary:false]]//.

//list// - a //[[CL:Glossary:list]]// of the //[[CL:Glossary:object|objects]]// read.

====Description====

**[[CL:Functions:read-delimited-list]]** reads //[[CL:Glossary:object|objects]]// from //input-stream// until the next character after an //[[CL:Glossary:object]]//'s representation (ignoring //[[CL:Glossary:whitespace]]// characters and comments) is //char//.

**[[CL:Functions:read-delimited-list]]** looks ahead at each step for the next non-//[[CL:Glossary:whitespace]]// //[[CL:Glossary:character]]// and peeks at it as if with **[[CL:Functions:peek-char]]**. If it is //char//, then the //[[CL:Glossary:character]]// is consumed and the //[[CL:Glossary:list]]// of //[[CL:Glossary:object|objects]]// is returned. If it is a //[[CL:Glossary:constituent]]// or //[[CL:Glossary:escape]]// //[[CL:Glossary:character]]//, then **[[CL:Functions:read]]** is used to read an //[[CL:Glossary:object]]//, which is added to the end of the //[[CL:Glossary:list]]//. If it is a //[[CL:Glossary:macro character]]//, its //[[CL:Glossary:reader macro function]]// is called; if the function returns a //[[CL:Glossary:value]]//, that //[[CL:Glossary:value]]// is added to the //[[CL:Glossary:list]]//. The peek-ahead process is then repeated.

If //recursive-p// is //[[CL:Glossary:true]]//, this call is expected to be embedded in a higher-level call to **[[CL:Functions:read]]** or a similar function.

It is an error to reach end-of-file during the operation of **[[CL:Functions:read-delimited-list]]**.

The consequences are undefined if //char// has a //[[CL:Glossary:syntax type]]// of //[[CL:Glossary:whitespace]]// in the //[[CL:Glossary:current readtable]]//.

====Examples==== <blockquote> (read-delimited-list #\\\rbracket) 1 2 3 4 5 6 \rbracket → (1 2 3 4 5 6) </blockquote>

Suppose you wanted \f{#{''a'' ''b'' ''c'' ''\ldots'' ''z''}} to read as a list of all pairs of the elements ''a'', ''b'', ''c'', ''\ldots'', ''z'', for example.

<blockquote> #{p q z a} reads as ((p q) (p z) (p a) (q z) (q a) (z a)) </blockquote> This can be done by specifying a macro-character definition for \f{#{} that does two things: reads in all the items up to the ''\''}, and constructs the pairs. **[[CL:Functions:read-delimited-list]]** performs the first task.

<blockquote> (defun |#{-reader| (stream char arg) (declare (ignore char arg)) (mapcon #'(lambda (x) (mapcar #'(lambda (y) (list (car x) y)) (cdr x))) (read-delimited-list #\\} stream t))) → |#{-reader|

(set-dispatch-macro-character #\# #\\{ #'|#{-reader|) → T (set-macro-character #\\} (get-macro-character #\\) **[[CL:Constant Variables:nil]]**)) </blockquote> Note that //[[CL:Glossary:true]]// is supplied for the //recursive-p// argument.

It is necessary here to give a definition to the character ''\''} as well to prevent it from being a constituent. If the line

<blockquote> (set-macro-character #\\} (get-macro-character #\\) **[[CL:Constant Variables:nil]]**)) </blockquote> shown above were not included, then the ''\''} in

<blockquote> #{ p q z a} </blockquote> would be considered a constituent character, part of the symbol named ''a\''}. This could be corrected by putting a space before the ''\''}, but it is better to call **[[CL:Functions:set-macro-character]]**.


Giving ''\''} the same definition as the standard definition of the character '')'' has the twin benefit of making it terminate tokens for use with **[[CL:Functions:read-delimited-list]]** and also making it invalid for use in any other context. Attempting to read a stray ''\''} will signal an error.

====Affected By====

**[[CL:Variables:*standard-input*]]**, **[[CL:Variables:*readtable*]]**, **[[CL:Variables:*terminal-io*]]**.

====Exceptional Situations====

None.

====See Also====

**[[CL:Functions:read]]**, **[[CL:Functions:peek-char]]**, **[[CL:Functions:read-char]]**, **[[CL:Functions:unread-char]]**.

====Notes====

**[[CL:Functions:read-delimited-list]]** is intended for use in implementing //[[CL:Glossary:reader macros]]//. Usually it is desirable for //char// to be a //[[CL:Glossary:terminating]]// //[[CL:Glossary:macro character]]// so that it can be used to delimit tokens; however, **[[CL:Functions:read-delimited-list]]** makes no attempt to alter the syntax specified for //char// by the current readtable. The caller must make any necessary changes to the readtable syntax explicitly.

