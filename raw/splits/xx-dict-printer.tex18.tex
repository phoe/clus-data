====== Variable *PRINT-CASE* ======

====Value Type====

One of the //[[CL:Glossary:symbols]]// **'':upcase''**, **'':downcase''**, or **'':capitalize''**.

====Initial Value====

The //[[CL:Glossary:symbol]]// **'':upcase''**.

====Description====

\Thevalueof{*print-case*} controls the case (upper, lower, or mixed) in which to print any uppercase characters in the names of //[[CL:Glossary:symbols]]// when vertical-bar syntax is not used.

**[[CL:Variables:*print-case*]]** has an effect at all times when \thevalueof{*print-escape*} is //[[CL:Glossary:false]]//. **[[CL:Variables:*print-case*]]** also has an effect when \thevalueof{*print-escape*} is //[[CL:Glossary:true]]// unless inside an escape context (i.e. unless between //[[CL:Glossary:vertical-bars]]// or after a //[[CL:Glossary:slash]]//).

====Examples====

<blockquote> (defun test-print-case () (dolist (*print-case* '(:upcase :downcase :capitalize)) (format t "~&~S ~S~ → TEST-PC ;; Although the choice of which characters to escape is specified by ;; *PRINT-CASE*, the choice of how to escape those characters ;; (i.e., whether single escapes or multiple escapes are used) ;; is implementation-dependent. The examples here show two of the ;; many valid ways in which escaping might appear. (test-print-case) ;Implementation A
▷ THIS-AND-THAT |And-something-elSE|
▷ this-and-that a\\n\\d-\\s\\o\\m\\e\\t\\h\\i\\n\\g-\\e\\lse
▷ This-And-That A\\n\\d-\\s\\o\\m\\e\\t\\h\\i\\n\\g-\\e\\lse → NIL (test-print-case) ;Implementation B
▷ THIS-AND-THAT |And-something-elSE|
▷ this-and-that a|nd-something-el|se
▷ This-And-That A|nd-something-el|se → NIL </blockquote>

====Affected By====

None.

====See Also====

**[[CL:Functions:write]]**

====Notes====

**[[CL:Functions:read]]** normally converts lowercase characters appearing in //[[CL:Glossary:symbols]]// to corresponding uppercase characters, so that internally print names normally contain only uppercase characters.

If **[[CL:Variables:*print-escape*]]** is //[[CL:Glossary:true]]//, lowercase characters in the //[[CL:Glossary:name]]// of a //[[CL:Glossary:symbol]]// are always printed in lowercase, and are preceded by a single escape character or enclosed by multiple escape characters; uppercase characters in the //[[CL:Glossary:name]]// of a //[[CL:Glossary:symbol]]// are printed in upper case, in lower case, or in mixed case so as to capitalize words, according to the value of **[[CL:Variables:*print-case*]]**. The convention for what constitutes a "word" is the same as for **[[CL:Functions:string-capitalize]]**.

\issue{PRINT-CASE-PRINT-ESCAPE-INTERACTION:VERTICAL-BAR-RULE-NO-UPCASE} \issue{JUN90-TRIVIAL-ISSUES:3} \issue{PRINT-CASE-BEHAVIOR:CLARIFY}
