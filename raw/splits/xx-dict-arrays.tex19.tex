====== Function ARRAY-TOTAL-SIZE ======

====Syntax====

**array-total-size** //array// → //size//

====Arguments and Values====

//array// - an //[[CL:Glossary:array]]//.

//size// - a non-negative //[[CL:Glossary:integer]]//.

====Description====

Returns the //[[CL:Glossary:array total size]]// of the //array//.

====Examples====

<blockquote> (array-total-size (make-array 4)) → 4 (array-total-size (make-array 4 :fill-pointer 2)) → 4 (array-total-size (make-array 0)) → 0 (array-total-size (make-array '(4 2))) → 8 (array-total-size (make-array '(4 0))) → 0 (array-total-size (make-array '())) → 1 </blockquote>

====Affected By====

None.

====Exceptional Situations====

Should signal an error of type **[[CL:Types:type-error]]** if its argument is not an //[[CL:Glossary:array]]//.

====See Also====

**[[CL:Functions:make-array]]**, **[[CL:Functions:array-dimensions]]**

====Notes====

If the //array// is a //[[CL:Glossary:vector]]// with a //[[CL:Glossary:fill pointer]]//, the //[[CL:Glossary:fill pointer]]// is ignored when calculating the //[[CL:Glossary:array total size]]//.

Since the product of no arguments is one, the //[[CL:Glossary:array total size]]// of a zero-dimensional //[[CL:Glossary:array]]// is one.

<blockquote> (array-total-size x) ≡ (apply #'* (array-dimensions x)) ≡ (reduce #'* (array-dimensions x)) </blockquote>

