====== Function UPGRADED-COMPLEX-PART-TYPE ======

====Syntax====

\DefunWithValues upgraded-complex-part-type {typespec ''&optional'' environment} {upgraded-typespec}

====Arguments and Values====

//typespec// - a //[[CL:Glossary:type specifier]]//.

//environment// - an //[[CL:Glossary:environment]]// //[[CL:Glossary:object]]//. The default is **[[CL:Constant Variables:nil]]**, denoting the //[[CL:Glossary:null lexical environment]]// and the and current //[[CL:Glossary:global environment]]//.

//upgraded-typespec// - a //[[CL:Glossary:type specifier]]//.

====Description====

**[[CL:Functions:upgraded-complex-part-type]]** returns the part type of the most specialized //[[CL:Glossary:complex]]// number representation that can hold parts of //[[CL:Glossary:type]]// //[[CL:Glossary:typespec]]//.

The //typespec// is a //[[CL:Glossary:subtype]]// of (and possibly //[[CL:Glossary:type equivalent]]// to) the //upgraded-typespec//.

The purpose of **[[CL:Functions:upgraded-complex-part-type]]** is to reveal how an implementation does its //[[CL:Glossary:upgrade|upgrading]]//.

====Examples====

None.

====Side Effects====

None.

====Affected By====

None.

====Exceptional Situations====

None.

====See Also====

**[[CL:Functions:complex]]** (//[[CL:Glossary:function]]// and //[[CL:Glossary:type]]//)

====Notes====

\issue{SUBTYPEP-ENVIRONMENT:ADD-ARG} \issue{ARRAY-TYPE-ELEMENT-TYPE-SEMANTICS:UNIFY-UPGRADING}
