====== Function INTERSECTION, NINTERSECTION ======

====Syntax====

**intersection {list-1 list-2** //\key} key test test-not// → //result-list// **nintersection {list-1 list-2** //\key} key test test-not// → //result-list//

====Arguments and Values====

//list-1// - a //[[CL:Glossary:proper list]]//.

//list-2// - a //[[CL:Glossary:proper list]]//.

//test// - a //[[CL:Glossary:designator]]// for a //[[CL:Glossary:function]]// of two //[[CL:Glossary:arguments]]// that returns a //[[CL:Glossary:generalized boolean]]//.

//test-not// - a //[[CL:Glossary:designator]]// for a //[[CL:Glossary:function]]// of two //[[CL:Glossary:arguments]]// that returns a //[[CL:Glossary:generalized boolean]]//.

//key// - a //[[CL:Glossary:designator]]// for a //[[CL:Glossary:function]]// of one argument, or **[[CL:Constant Variables:nil]]**.

//result-list// - a //[[CL:Glossary:list]]//.

====Description====

**[[CL:Functions:intersection]]** and **[[CL:Functions:nintersection]]** return a //[[CL:Glossary:list]]// that contains every element that occurs in both //list-1// and //list-2//.

**[[CL:Functions:nintersection]]** is the destructive version of **[[CL:Functions:intersection]]**. It performs the same operation, but may destroy //list-1// using its cells to construct the result.

//list-2// is not destroyed.

The intersection operation is described as follows. For all possible ordered pairs consisting of one //[[CL:Glossary:element]]// from //list-1// and one //[[CL:Glossary:element]]// from //list-2//, **'':test''** or **'':test-not''** are used to determine whether they //[[CL:Glossary:satisfy the test]]//. The first argument to the **'':test''** or **'':test-not''** function is an element of //list-1//; the second argument is an element of //list-2//. If **'':test''** or **'':test-not''** is not supplied, **[[CL:Functions:eql]]** is used. It is an error if **'':test''** and **'':test-not''** are supplied in the same function call.

If **'':key''** is supplied (and not **[[CL:Constant Variables:nil]]**), it is used to extract the part to be tested from the //list// element. The argument to the **'':key''** function is an element of either //list-1// or //list-2//; the **'':key''** function typically returns part of the supplied element. If **'':key''** is not supplied or **[[CL:Constant Variables:nil]]**, the //list-1// and //list-2// elements are used.

For every pair that //[[CL:Glossary:satifies the test]]//, exactly one of the two elements of the pair will be put in the result. No element from either //[[CL:Glossary:list]]// appears in the result that does not //[[CL:Glossary:satisfy the test]]// for an element from the other //[[CL:Glossary:list]]//. If one of the //[[CL:Glossary:lists]]// contains duplicate elements, there may be duplication in the result.

There is no guarantee that the order of elements in the result will reflect the ordering of the arguments in any particular way. The result //[[CL:Glossary:list]]// may share cells with, or be **[[CL:Functions:eq]]** to, either //list-1// or //list-2// if appropriate.

====Examples====

<blockquote> ([[CL:Macros:defparameter]] list1 (list 1 1 2 3 4 a b c "A" "B" "C" "d") list2 (list 1 4 5 b c d "a" "B" "c" "D")) → (1 4 5 B C D "a" "B" "c" "D") (intersection list1 list2) → (C B 4 1 1) (intersection list1 list2 :test 'equal) → ("B" C B 4 1 1) (intersection list1 list2 :test #'equalp) → ("d" "C" "B" "A" C B 4 1 1) (nintersection list1 list2) → (1 1 4 B C) list1 → //[[CL:Glossary:implementation-dependent]]// ;//e.g.// (1 1 4 B C) list2 → //[[CL:Glossary:implementation-dependent]]// ;//e.g.// (1 4 5 B C D "a" "B" "c" "D") ([[CL:Macros:defparameter]] list1 (copy-list '((1 . 2) (2 . 3) (3 . 4) (4 . 5)))) → ((1 . 2) (2 . 3) (3 . 4) (4 . 5)) ([[CL:Macros:defparameter]] list2 (copy-list '((1 . 3) (2 . 4) (3 . 6) (4 . 8)))) → ((1 . 3) (2 . 4) (3 . 6) (4 . 8)) (nintersection list1 list2 :key #'cdr) → ((2 . 3) (3 . 4)) list1 → //[[CL:Glossary:implementation-dependent]]// ;//e.g.// ((1 . 2) (2 . 3) (3 . 4)) list2 → //[[CL:Glossary:implementation-dependent]]// ;//e.g.// ((1 . 3) (2 . 4) (3 . 6) (4 . 8)) </blockquote>

====Side Effects====

**[[CL:Functions:nintersection]]** can modify //list-1//,

but not //list-2//.

====Affected By====

None.

====Exceptional Situations====

Should be prepared to signal an error of type type-error if //list-1// and //list-2// are not //[[CL:Glossary:proper lists]]//.

====See Also====

**[[CL:Functions:union]]**,

{\secref\ConstantModification},

{\secref\TraversalRules}

====Notes====

The **'':test-not''** parameter is deprecated.

Since the **[[CL:Functions:nintersection]]** side effect is not required, it should not be used in for-effect-only positions in portable code.

\issue{NINTERSECTION-DESTRUCTION:REVERT} \issue{REMF-DESTRUCTION-UNSPECIFIED:X3J13-MAR-89} \issue{REMF-DESTRUCTION-UNSPECIFIED:X3J13-MAR-89} \issue{NINTERSECTION-DESTRUCTION} \issue{CONSTANT-MODIFICATION:DISALLOW} \issue{MAPPING-DESTRUCTIVE-INTERACTION:EXPLICITLY-VAGUE} \issue{TEST-NOT-IF-NOT:FLUSH-ALL} \issue{REMF-DESTRUCTION-UNSPECIFIED:X3J13-MAR-89}
