====== System Class NUMBER ======

====Class Precedence List==== **[[CL:Types:number]]**, **[[CL:Types:t]]**

====Description====


The type **[[CL:Types:number]]** contains //[[CL:Glossary:object|objects]]// which represent mathematical numbers.

The //[[CL:Glossary:types]]// **[[CL:Types:real]]** and **[[CL:Types:complex]]** are //[[CL:Glossary:disjoint]]// //[[CL:Glossary:subtypes]]// of **[[CL:Types:number]]**.

The //[[CL:Glossary:function]]// **[[CL:Functions:=]]** tests for numerical equality. The //[[CL:Glossary:function]]// **[[CL:Functions:eql]]**, when its arguments are both //[[CL:Glossary:numbers]]//, tests that they have both the same //[[CL:Glossary:type]]// and numerical value.

Two //[[CL:Glossary:numbers]]// that are the //[[CL:Glossary:same]]// under **[[CL:Functions:eql]]** or **[[CL:Functions:=]]** are not necessarily the //[[CL:Glossary:same]]// under **[[CL:Functions:eq]]**.

====Notes====

\clisp\ differs from mathematics on some naming issues. In mathematics, the set of real numbers is traditionally described as a subset of the complex numbers, but in \clisp, the type **[[CL:Types:real]]** and the type **[[CL:Types:complex]]** are disjoint. The \clisp\ type which includes all mathematical complex numbers is called **[[CL:Types:number]]**. The reasons for these differences include historical precedent, compatibility with most other popular computer languages, and various issues of time and space efficiency.

\issue{REAL-NUMBER-TYPE:X3J13-MAR-89}
