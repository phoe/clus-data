====== Special Form SETQ ======

====Syntax====

\DefspecWithValues setq {\stardown{pair}} {result}

\auxbnf{pair}{var form}

====Arguments and Values====

//var// - a //[[CL:Glossary:symbol]]// naming a //[[CL:Glossary:variable]]// other than a //[[CL:Glossary:constant variable]]//.

//form// - a //[[CL:Glossary:form]]//.

//result// - the //[[CL:Glossary:primary value]]// of the last //form//, or **[[CL:Constant Variables:nil]]** if no //pairs// were supplied.

====Description====

Assigns values to //[[CL:Glossary:variables]]//.

''(setq ''var1'' ''form1'' ''var2'' ''form2'' ...)'' is the simple variable assignment statement of Lisp. First //form1// is evaluated and the result is stored in the variable //var1//, then //form2// is evaluated and the result stored in //var2//, and so forth. \specref{setq} may be used for assignment of both lexical and dynamic variables.

If any //var// refers to a //[[CL:Glossary:binding]]// made by \specref{symbol-macrolet}, then that //var// is treated as if **[[CL:Macros:setf]]** (not \specref{setq}) had been used.

====Examples====

<blockquote> ;; A simple use of SETQ to establish values for variables. ([[CL:Macros:defparameter]] a 1 b 2 c 3) → 3 a → 1 b → 2 c → 3

;; Use of SETQ to update values by sequential assignment. ([[CL:Macros:defparameter]] a (1+ b) b (1+ a) c (+ a b)) → 7 a → 3 b → 4 c → 7

;; This illustrates the use of SETQ on a symbol macro. (let ((x (list 10 20 30))) (symbol-macrolet ((y (car x)) (z (cadr x))) ([[CL:Macros:defparameter]] y (1+ z) z (1+ y)) (list x y z))) → ((21 22 30) 21 22) </blockquote>

====Side Effects====

The //[[CL:Glossary:primary value]]// of each //form// is assigned to the corresponding //var//.

====Affected By====

None.

====Exceptional Situations====

None.

====See Also====

**[[CL:Macros:psetq]]**, **[[CL:Functions:set]]**, **[[CL:Macros:setf]]**

====Notes====

None.

\issue{SYMBOL-MACROLET-SEMANTICS:SPECIAL-FORM}
