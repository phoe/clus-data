====== Function DISASSEMBLE ======

====Syntax====

**disassemble** //fn// → //**[[CL:Constant Variables:nil]]**//

====Arguments and Values====

//fn// - an //[[CL:Glossary:extended function designator]]// or a //[[CL:Glossary:lambda expression]]//.

====Description====

The //[[CL:Glossary:function]]// **[[CL:Functions:disassemble]]** is a debugging aid that composes symbolic instructions or expressions in some //[[CL:Glossary:implementation-dependent]]// language which represent the code used to produce the //[[CL:Glossary:function]]// which is or is named by the argument //fn//. The result is displayed to //[[CL:Glossary:standard output]]// in an //[[CL:Glossary:implementation-dependent]]// format.

If //fn// is a //[[CL:Glossary:lambda expression]]// or //[[CL:Glossary:interpreted function]]//, it is compiled first and the result is disassembled.

If the //fn// //[[CL:Glossary:designator]]// is a //[[CL:Glossary:function name]]//, the //[[CL:Glossary:function]]// that it //[[CL:Glossary:names]]// is disassembled.

(If that //[[CL:Glossary:function]]// is an //[[CL:Glossary:interpreted function]]//, it is first compiled but the result of this implicit compilation is not installed.)

====Examples====

<blockquote> (defun f (a) (1+ a)) → F (eq (symbol-function 'f) (progn (disassemble 'f) (symbol-function 'f))) → //[[CL:Glossary:true]]// </blockquote>

====Side Effects====

None.

====Affected By====

**[[CL:Variables:*standard-output*]]**.

====Exceptional Situations====

Should signal an error of type type-error if //fn// is not an //[[CL:Glossary:extended function designator]]// or a //[[CL:Glossary:lambda expression]]//.

====See Also====

None.

====Notes====

None.

\issue{FUNCTION-NAME:LARGE} \issue{DISASSEMBLE-SIDE-EFFECT:DO-NOT-INSTALL} \issue{DISASSEMBLE-SIDE-EFFECT:DO-NOT-INSTALL} \issue{DOCUMENTATION-FUNCTION-TANGLED:REQUIRE-ARGUMENT}
