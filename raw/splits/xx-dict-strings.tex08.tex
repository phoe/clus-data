\begincom{string-upcase, string-downcase, string-capitalize, nstring-upcase, nstring-downcase, nstring-capitalize}\ftype{Function}

====Syntax====

\DefunMultiWithValues {string ''&key'' start end} {cased-string} {string-upcase string-downcase string-capitalize}

\DefunMultiWithValues {string ''&key'' start end} {string} {nstring-upcase nstring-downcase nstring-capitalize}

====Arguments and Values====

//string// - a //[[CL:Glossary:string designator]]//. For **[[CL:Functions:nstring-upcase]]**, **[[CL:Functions:nstring-downcase]]**, and **[[CL:Functions:nstring-capitalize]]**, the //string// //[[CL:Glossary:designator]]// must be a //[[CL:Glossary:string]]//.



//start//, //end// - //[[CL:Glossary:bounding index designators]]// of //string//. \Defaults{//start// and //end//}{''0'' and **[[CL:Constant Variables:nil]]**}

//cased-string// - a //[[CL:Glossary:string]]//.

====Description====

**[[CL:Functions:string-upcase]]**, **[[CL:Functions:string-downcase]]**, **[[CL:Functions:string-capitalize]]**, **[[CL:Functions:nstring-upcase]]**, **[[CL:Functions:nstring-downcase]]**, **[[CL:Functions:nstring-capitalize]]** change the case of the subsequence of //string// //[[CL:Glossary:bounded]]// by //start// and //end// as follows:

\beginlist \itemitem{\bf string-upcase}

**[[CL:Functions:string-upcase]]** returns a //[[CL:Glossary:string]]// just like //string// with all lowercase characters replaced by the corresponding uppercase characters. More precisely, each character of the result //[[CL:Glossary:string]]// is produced by applying the function **[[CL:Functions:char-upcase]]** to the corresponding character of //string//.

\itemitem{\bf string-downcase}

**[[CL:Functions:string-downcase]]** is like **[[CL:Functions:string-upcase]]** except that all uppercase characters are replaced by the corresponding lowercase characters (using **[[CL:Functions:char-downcase]]**).

\itemitem{\bf string-capitalize}

**[[CL:Functions:string-capitalize]]** produces a copy of //string// such that, for every word in the copy, the first //[[CL:Glossary:character]]// of the "word," if it has //[[CL:Glossary:case]]//, is //[[CL:Glossary:uppercase]]// and any other //[[CL:Glossary:characters]]// with //[[CL:Glossary:case]]// in the word are //[[CL:Glossary:lowercase]]//. For the purposes of **[[CL:Functions:string-capitalize]]**, a "word" is defined to be a

consecutive subsequence consisting of //[[CL:Glossary:alphanumeric]]// //[[CL:Glossary:characters]]//, delimited at each end either by a non-//[[CL:Glossary:alphanumeric]]// //[[CL:Glossary:character]]// or by an end of the //[[CL:Glossary:string]]//.

\itemitem{\bf nstring-upcase, nstring-downcase, nstring-capitalize }

**[[CL:Functions:nstring-upcase]]**, **[[CL:Functions:nstring-downcase]]**, and **[[CL:Functions:nstring-capitalize]]** are identical to **[[CL:Functions:string-upcase]]**, **[[CL:Functions:string-downcase]]**, and **[[CL:Functions:string-capitalize]]** respectively except that they modify //string//. \endlist

For **[[CL:Functions:string-upcase]]**, **[[CL:Functions:string-downcase]]**, and **[[CL:Functions:string-capitalize]]**, //string// is not modified. However, if no characters in //string// require conversion, the result may be either //string// or a copy of it, at the implementation's discretion.

====Examples==== <blockquote> (string-upcase "abcde") → "ABCDE" (string-upcase "Dr. Livingston, I presume?") → "DR. LIVINGSTON, I PRESUME?" (string-upcase "Dr. Livingston, I presume?" :start 6 :end 10) → "Dr. LiVINGston, I presume?" (string-downcase "Dr. Livingston, I presume?") → "dr. livingston, i presume?"

(string-capitalize "elm 13c arthur;fig don't") → "Elm 13c Arthur;Fig Don'T" (string-capitalize " hello ") → " Hello " (string-capitalize "occlUDeD cASEmenTs FOreSTAll iNADVertent DEFenestraTION") → "Occluded Casements Forestall Inadvertent Defenestration" (string-capitalize 'kludgy-hash-search) → "Kludgy-Hash-Search" (string-capitalize "DON'T!") → "Don'T!" ;not "Don't!" (string-capitalize "pipe 13a, foo16c") → "Pipe 13a, Foo16c"

([[CL:Macros:defparameter]] str (copy-seq "0123ABCD890a")) → "0123ABCD890a" (nstring-downcase str :start 5 :end 7) → "0123AbcD890a" str → "0123AbcD890a" </blockquote>

====Side Effects====

**[[CL:Functions:nstring-upcase]]**, **[[CL:Functions:nstring-downcase]]**, and **[[CL:Functions:nstring-capitalize]]** modify //string// as appropriate rather than constructing a new //[[CL:Glossary:string]]//.

====Affected By====

None.

====Exceptional Situations====

None.

====See Also====

**[[CL:Functions:char-upcase]]**, **[[CL:Functions:char-downcase]]**

====Notes====

The result is always of the same length as //string//.

\issue{STRING-COERCION:MAKE-CONSISTENT} \issue{SUBSEQ-OUT-OF-BOUNDS} \issue{RANGE-OF-START-AND-END-PARAMETERS:INTEGER-AND-INTEGER-NIL}
