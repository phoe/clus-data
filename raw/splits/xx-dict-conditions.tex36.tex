====== Macro RESTART-BIND ======

====Syntax====

\DefmacWithValuesNewline restart-bind {\paren{\curly{\paren{name function \stardown{key-val-pair}}}} \starparam{form}} {\starparam{result}}

\auxbnf{key-val-pair}{**'':interactive-function''** {interactive-function} | \CR **'':report-function''** {report-function} | \CR **'':test-function''** {test-function}}

====Arguments and Values====

//name// - a //[[CL:Glossary:symbol]]//; \noeval.

//function// - a //[[CL:Glossary:form]]//; \eval.

//forms// - an //[[CL:Glossary:implicit progn]]//.

//interactive-function// - a //[[CL:Glossary:form]]//; \eval.

//report-function// - a //[[CL:Glossary:form]]//; \eval.

//test-function// - a //[[CL:Glossary:form]]//; \eval.

//results// - the //[[CL:Glossary:values]]// returned by the //[[CL:Glossary:forms]]//.

====Description====

**[[CL:Macros:restart-bind]]** executes the body of //forms// in a //[[CL:Glossary:dynamic environment]]// where //[[CL:Glossary:restarts]]// with the given //names// are in effect.

If a //name// is **[[CL:Constant Variables:nil]]**, it indicates an anonymous restart; if a //name// is a //[[CL:Glossary:non-nil]]// //[[CL:Glossary:symbol]]//, it indicates a named restart.

The //function//, //interactive-function//, and //report-function// are unconditionally evaluated in the current lexical and dynamic environment prior to evaluation of the body. Each of these //[[CL:Glossary:forms]]// must evaluate to a //[[CL:Glossary:function]]//.

If **[[CL:Functions:invoke-restart]]** is done on that restart, the //[[CL:Glossary:function]]// which resulted from evaluating //function// is called, in the //[[CL:Glossary:dynamic environment]]// of the **[[CL:Functions:invoke-restart]]**, with the //[[CL:Glossary:arguments]]// given to **[[CL:Functions:invoke-restart]]**. The //[[CL:Glossary:function]]// may either perform a non-local transfer of control or may return normally.

If the restart is invoked interactively from the debugger (using **[[CL:Functions:invoke-restart-interactively]]**), the arguments are defaulted by calling the //[[CL:Glossary:function]]// which resulted from evaluating //interactive-function//. That //[[CL:Glossary:function]]// may optionally prompt interactively on //[[CL:Glossary:query I/O]]//, and should return a //[[CL:Glossary:list]]// of arguments to be used by **[[CL:Functions:invoke-restart-interactively]]** when invoking the restart.

If a restart is invoked interactively but no //interactive-function// is used, then an argument list of **[[CL:Constant Variables:nil]]** is used. In that case, the //[[CL:Glossary:function]]// must be compatible with an empty argument list.

If the restart is presented interactively (//e.g.// by the debugger), the presentation is done by calling the //[[CL:Glossary:function]]// which resulted from evaluating //report-function//. This //[[CL:Glossary:function]]// must be a //[[CL:Glossary:function]]// of one argument, a //[[CL:Glossary:stream]]//. It is expected to print a description of the action that the restart takes to that //[[CL:Glossary:stream]]//. This //[[CL:Glossary:function]]// is called any time the restart is printed while **[[CL:Variables:*print-escape*]]** is **[[CL:Constant Variables:nil]]**.

In the case of interactive invocation, the result is dependent on the value of **'':interactive-function''** as follows.

\beginlist \itemitem{**'':interactive-function''**}

//Value// is evaluated in the current lexical environment and should return a //[[CL:Glossary:function]]// of no arguments which constructs a //[[CL:Glossary:list]]// of arguments to be used by **[[CL:Functions:invoke-restart-interactively]]** when invoking this restart. The //[[CL:Glossary:function]]// may prompt interactively using //[[CL:Glossary:query I/O]]// if necessary.

\itemitem{**'':report-function''**}

//Value// is evaluated in the current lexical environment and should return a //[[CL:Glossary:function]]// of one argument, a //[[CL:Glossary:stream]]//, which prints on the //[[CL:Glossary:stream]]// a summary of the action that this restart takes. This //[[CL:Glossary:function]]// is called whenever the restart is reported (printed while **[[CL:Variables:*print-escape*]]** is **[[CL:Constant Variables:nil]]**).

If no **'':report-function''** option is provided, the manner in which the //[[CL:Glossary:restart]]// is reported is //[[CL:Glossary:implementation-dependent]]//.

\itemitem{**'':test-function''**}

//Value// is evaluated in the current lexical environment and should return a //[[CL:Glossary:function]]// of one argument, a //[[CL:Glossary:condition]]//, which returns //[[CL:Glossary:true]]// if the restart is to be considered visible.

\endlist

====Side Effects====

None.

====Affected By====

**[[CL:Variables:*query-io*]]**.

====Exceptional Situations====

None.

====See Also====

**[[CL:Macros:restart-case]]**, **[[CL:Macros:with-simple-restart]]**

====Notes====

**[[CL:Macros:restart-bind]]** is primarily intended to be used to implement **[[CL:Macros:restart-case]]** and might be useful in implementing other macros. Programmers who are uncertain about whether to use **[[CL:Macros:restart-case]]** or **[[CL:Macros:restart-bind]]** should prefer **[[CL:Macros:restart-case]]** for the cases where it is powerful enough, using **[[CL:Macros:restart-bind]]** only in cases where its full generality is really needed.


\issue{CONDITION-RESTARTS:PERMIT-ASSOCIATION}
