====== Accessor GETHASH ======

====Syntax====

**gethash {key hash-table** //\opt} default// → //value, present-p// (**setf** (**gethash {key hash-table** //\opt} default//) //new-value//)

====Arguments and Values====

//key// - an //[[CL:Glossary:object]]//.

//hash-table// - a //[[CL:Glossary:hash table]]//.

//default// - an //[[CL:Glossary:object]]//. The default is **[[CL:Constant Variables:nil]]**.

//value// - an //[[CL:Glossary:object]]//.

//present-p// - a //[[CL:Glossary:generalized boolean]]//.

====Description====

//Value// is the //[[CL:Glossary:object]]// in //hash-table// whose //[[CL:Glossary:key]]// is the //[[CL:Glossary:same]]// as //key// under the //hash-table//'s equivalence test. If there is no such entry, //value// is the //default//.

//Present-p// is //[[CL:Glossary:true]]// if an entry is found; otherwise, it is //[[CL:Glossary:false]]//.

**[[CL:Macros:setf]]** may be used with **[[CL:Functions:gethash]]** to modify the //[[CL:Glossary:value]]// associated with a given //[[CL:Glossary:key]]//, or to add a new entry.

When a **[[CL:Functions:gethash]]** //[[CL:Glossary:form]]// is used as a **[[CL:Macros:setf]]** //place//, any //default// which is supplied is evaluated according to normal left-to-right evaluation rules, but its //[[CL:Glossary:value]]// is ignored.

====Examples====

<blockquote> ([[CL:Macros:defparameter]] table (make-hash-table)) → #<HASH-TABLE EQL 0/120 32206334> (gethash 1 table) → NIL, //[[CL:Glossary:false]]// (gethash 1 table 2) → 2, //[[CL:Glossary:false]]// ([[CL:Macros:setf]] (gethash 1 table) "one") → "one" ([[CL:Macros:setf]] (gethash 2 table "two") "two") → "two" (gethash 1 table) → "one", //[[CL:Glossary:true]]// (gethash 2 table) → "two", //[[CL:Glossary:true]]// (gethash nil table) → NIL, //[[CL:Glossary:false]]// ([[CL:Macros:setf]] (gethash nil table) nil) → NIL (gethash nil table) → NIL, //[[CL:Glossary:true]]// (defvar *counters* (make-hash-table)) → *COUNTERS* (gethash 'foo *counters*) → NIL, //[[CL:Glossary:false]]// (gethash 'foo *counters* 0) → 0, //[[CL:Glossary:false]]// (defmacro how-many (obj) `(values (gethash ,obj *counters* 0))) → HOW-MANY (defun count-it (obj) (incf (how-many obj))) → COUNT-IT (dolist (x '(bar foo foo bar bar baz)) (count-it x)) (how-many 'foo) → 2 (how-many 'bar) → 3 (how-many 'quux) → 0 </blockquote>

====Side Effects====

None.

====Affected By====

None.

====Exceptional Situations====

None.

====See Also====

**[[CL:Functions:remhash]]**

====Notes====

The //[[CL:Glossary:secondary value]]//, //present-p//, can be used to distinguish the absence of an entry from the presence of an entry that has a value of //default//.

\issue{SETF-GET-DEFAULT:EVALUATED-BUT-IGNORED}
