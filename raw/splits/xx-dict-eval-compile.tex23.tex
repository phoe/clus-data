====== Declaration INLINE, NOTINLINE ======

====Syntax====

{\tt (inline \starparam{function-name})}

{\tt (notinline \starparam{function-name})}

====Arguments====

//function-name// - a //[[CL:Glossary:function name]]//.

====Valid Context====

//[[CL:Glossary:declaration]]// or //[[CL:Glossary:proclamation]]//

====Binding Types Affected====

//[[CL:Glossary:function]]//

====Description====

\declref{inline} specifies that it is desirable for the compiler to produce inline calls to the //[[CL:Glossary:functions]]// named by //function-names//; that is, the code for a specified //function-name//

should be integrated into the calling routine, appearing "in line" in place of a procedure call.

A compiler is free to ignore this declaration. \declref{inline} declarations never apply to variable //[[CL:Glossary:bindings]]//.

If one of the //[[CL:Glossary:functions]]// mentioned has a lexically apparent local definition (as made by \specref{flet} or \specref{labels}), then the declaration applies to that local definition and not to the global function definition.

While no //[[CL:Glossary:conforming implementation]]// is required to perform inline expansion of user-defined functions, those //[[CL:Glossary:implementations]]// that do attempt to recognize the following paradigm:

To define a //[[CL:Glossary:function]]// ''f'' that is not \declref{inline} by default but for which ''(declare (inline f))'' will make //f// be locally inlined, the proper definition sequence is:

<blockquote> (declaim (inline f)) (defun f ...) (declaim (notinline f)) </blockquote>

The \declref{inline} proclamation preceding the **[[CL:Macros:defun]]** //[[CL:Glossary:form]]// ensures that the //[[CL:Glossary:compiler]]// has the opportunity save the information necessary for inline expansion, and the \declref{notinline} proclamation following the **[[CL:Macros:defun]]** //[[CL:Glossary:form]]// prevents ''f'' from being expanded inline everywhere.

\declref{notinline} specifies that it is

undesirable to compile the //[[CL:Glossary:functions]]// named by //function-names// in-line.

A compiler is not free to ignore this declaration;

calls to the specified functions must be implemented as out-of-line subroutine calls.

If one of the //[[CL:Glossary:functions]]// mentioned has a lexically apparent local definition (as made by \specref{flet} or \specref{labels}), then the declaration applies to that local definition and not to the global function definition.

In the presence of a //[[CL:Glossary:compiler macro]]// definition for //function-name//, a \declref{notinline} declaration prevents that

//[[CL:Glossary:compiler macro]]// from being used.

An \declref{inline} declaration may be used to encourage use of //[[CL:Glossary:compiler macro]]// definitions. \declref{inline} and \declref{notinline} declarations otherwise have no effect when the lexically visible definition of //function-name// is a //[[CL:Glossary:macro]]// definition.

\declref{inline} and \declref{notinline} declarations can be //[[CL:Glossary:free declarations]]// or //[[CL:Glossary:bound declarations]]//. \declref{inline} and \declref{notinline} declarations of functions that appear before the body of a \specref{flet} or \specref{labels}

//[[CL:Glossary:form]]// that defines that function are //[[CL:Glossary:bound declarations]]//. Such declarations in other contexts are //[[CL:Glossary:free declarations]]//.

====Examples====

<blockquote> ;; The globally defined function DISPATCH should be open-coded, ;; if the implementation supports inlining, unless a NOTINLINE ;; declaration overrides this effect. (declaim (inline dispatch)) (defun dispatch (x) (funcall (get (car x) 'dispatch) x)) ;; Here is an example where inlining would be encouraged. (defun top-level-1 () (dispatch (read-command))) ;; Here is an example where inlining would be prohibited. (defun top-level-2 () (declare (notinline dispatch)) (dispatch (read-command))) ;; Here is an example where inlining would be prohibited. (declaim (notinline dispatch)) (defun top-level-3 () (dispatch (read-command))) ;; Here is an example where inlining would be encouraged. (defun top-level-4 () (declare (inline dispatch)) (dispatch (read-command))) </blockquote>

====See Also====

\misc{declare}, **[[CL:Macros:declaim]]**, **[[CL:Functions:proclaim]]**

\issue{MACRO-DECLARATIONS:MAKE-EXPLICIT} \issue{KMP-COMMENTS-ON-SANDRA-COMMENTS:X3J13-MAR-92} \issue{WITH-ADDED-METHODS:DELETE} \issue{GENERIC-FLET-POORLY-DESIGNED:DELETE} \issue{FUNCTION-NAME:LARGE} \issue{ALLOW-LOCAL-INLINE:INLINE-NOTINLINE} \issue{FUNCTION-NAME:LARGE}
