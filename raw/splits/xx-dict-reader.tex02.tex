====== Function COPY-READTABLE ======

====Syntax====

**{copy-readtable}** //''&optional'' from-readtable to-readtable// → //readtable//

====Arguments and Values====

//from-readtable// - a //[[CL:Glossary:readtable designator]]//. The default is the //[[CL:Glossary:current readtable]]//.

//to-readtable// - a //[[CL:Glossary:readtable]]// or **[[CL:Constant Variables:nil]]**. The default is **[[CL:Constant Variables:nil]]**.

//readtable// - the //to-readtable// if it is //[[CL:Glossary:non-nil]]//, or else a //[[CL:Glossary:fresh]]// //[[CL:Glossary:readtable]]//.

====Description====

**[[CL:Functions:copy-readtable]]** copies //from-readtable//.

If //to-readtable// is **[[CL:Constant Variables:nil]]**, a new //[[CL:Glossary:readtable]]// is created and returned. Otherwise the //[[CL:Glossary:readtable]]// specified by //to-readtable// is modified and returned.

**[[CL:Functions:copy-readtable]]** copies the setting of **[[CL:Functions:readtable-case]]**.

====Examples====

<blockquote> ([[CL:Macros:defparameter]] zvar 123) → 123 (set-syntax-from-char #\\z #\\' ([[CL:Macros:defparameter]] table2 (copy-readtable))) → T zvar → 123 (copy-readtable table2 *readtable*) → #<READTABLE 614000277> zvar → VAR ([[CL:Macros:defparameter]] *readtable* (copy-readtable)) → #<READTABLE 46210223> zvar → VAR ([[CL:Macros:defparameter]] *readtable* (copy-readtable nil)) → #<READTABLE 46302670> zvar → 123 </blockquote>

====Affected By====

None.

====Exceptional Situations====

None.

====See Also====

**[[CL:Types:readtable]]**, **[[CL:Variables:*readtable*]]**

====Notes====

<blockquote> ([[CL:Macros:defparameter]] *readtable* (copy-readtable nil)) </blockquote> restores the input syntax to standard \clisp\ syntax, even if the //[[CL:Glossary:initial readtable]]// has been clobbered (assuming it is not so badly clobbered that you cannot type in the above expression).

On the other hand,

<blockquote> ([[CL:Macros:defparameter]] *readtable* (copy-readtable)) </blockquote> replaces the current //[[CL:Glossary:readtable]]// with a copy of itself. This is useful if you want to save a copy of a readtable for later use, protected from alteration in the meantime. It is also useful if you want to locally bind the readtable to a copy of itself, as in:

<blockquote> (let ((*readtable* (copy-readtable))) ...) </blockquote>

\issue{READ-CASE-SENSITIVITY:READTABLE-KEYWORDS}
