

\beginSubsection{Syntax of Logical Pathname Namestrings}
\DefineSection{LogPathNamestrings}

\issue{PATHNAME-LOGICAL:ADD}
The syntax of a //[[CL:Glossary:logical pathname]]// //[[CL:Glossary:namestring]]// is as follows.

(Note that unlike many notational descriptions in this document,
 this is a syntactic description of character sequences,
 not a structural description of //[[CL:Glossary:objects]]//.)

\auxbnf{logical-pathname}%
{\brac{\down{host} //host-marker//} \CR
 \brac{\down{//relative-directory-marker//}}
 \star{\curly{\down{directory} //directory-marker//}} \CR
 \brac{\down{name}} 
 \brac{//type-marker// \down{type} 
 \brac{//version-marker// \down{version}}}}

\auxbnf{host}{\down{word}}
\auxbnf{directory}{\down{word} | \down{wildcard-word} | \down{wild-inferiors-word}}
\auxbnf{name}{\down{word} | \down{wildcard-word}}
\auxbnf{type}{\down{word} | \down{wildcard-word}}
\auxbnf{version}{\down{pos-int} | //newest-word// | //wildcard-version//}

//host-marker//---a //[[CL:Glossary:colon]]//.

//relative-directory-marker//---a //[[CL:Glossary:semicolon]]//.

//directory-marker//---a //[[CL:Glossary:semicolon]]//.

//type-marker//---a //[[CL:Glossary:dot]]//.

//version-marker//---a //[[CL:Glossary:dot]]//.

//wild-inferiors-word//---The two character sequence ``\f{**}'' (two //[[CL:Glossary:asterisks]]//).

//newest-word//---The six character sequence ``\f{newest}'' 
		   or the six character sequence ``\f{NEWEST}''.

//wildcard-version//---an //[[CL:Glossary:asterisk]]//.

//wildcard-word//---one or more //[[CL:Glossary:asterisks]]//, uppercase letters,
   digits, and hyphens, including at least one //[[CL:Glossary:asterisk]]//, 
   with no two //[[CL:Glossary:asterisks]]// adjacent.

//word//---one or more uppercase letters, digits, and hyphens.

//pos-int//---a positive //[[CL:Glossary:integer]]//.

\beginsubsubsection{Additional Information about Parsing Logical Pathname Namestrings}

\beginsubsubsubsection{The Host part of a Logical Pathname Namestring}

The //host// must have been defined as a //[[CL:Glossary:logical pathname]]// host;
this can be done by using \macref{setf} of **[[CL:Functions:logical-pathname-translations]]**.

The //[[CL:Glossary:logical pathname]]// host name \f{"SYS"} is reserved for the implementation.
The existence and meaning of \f{SYS:} //[[CL:Glossary:logical pathnames]]// 
is //[[CL:Glossary:implementation-defined]]//.
 
\endsubsubsubsection%{The Host part of a Logical Pathname Namestring}

\beginsubsubsubsection{The Device part of a Logical Pathname Namestring}
 
There is no syntax for a //[[CL:Glossary:logical pathname]]// device since
the device component of a //[[CL:Glossary:logical pathname]]// is always **'':unspecific''**;
\seesection\LogicalPathCompUnspecific.

\endsubsubsubsection%{The Device part of a Logical Pathname Namestring}

\beginsubsubsubsection{The Directory part of a Logical Pathname Namestring}

If a //relative-directory-marker// precedes the //directories//,
the directory component parsed is as //[[CL:Glossary:relative]]//;
otherwise, the directory component is parsed as //[[CL:Glossary:absolute]]//.

If a //wild-inferiors-marker// is specified,
it parses into **'':wild-inferiors''**.
 
\endsubsubsubsection%{The Directory part of a Logical Pathname Namestring}

\beginsubsubsubsection{The Type part of a Logical Pathname Namestring}

The //type// of a //[[CL:Glossary:logical pathname]]// for a //[[CL:Glossary:source file]]//
is \f{"LISP"}.   This should be translated into whatever type is 
appropriate in a physical pathname.
 
\endsubsubsubsection%{The Type part of a Logical Pathname Namestring}

\beginsubsubsubsection{The Version part of a Logical Pathname Namestring}

Some //[[CL:Glossary:file systems]]// do not have //versions//. 
//[[CL:Glossary:Logical pathname]]// translation to such a //[[CL:Glossary:file system]]//
ignores the //version//.
This implies that a program cannot rely on being able to store
more than one version of a file named by a //[[CL:Glossary:logical pathname]]//.

If a //wildcard-version// is specified,
it parses into **'':wild''**.

\endsubsubsubsection%{The Version part of a Logical Pathname Namestring}

\beginsubsubsubsection{Wildcard Words in a Logical Pathname Namestring}

Each //[[CL:Glossary:asterisk]]// in a //wildcard-word// matches a sequence of 
zero or more characters.  The //wildcard-word// ``\f{*}'' 
parses into **'':wild''**; other //[[CL:Glossary:wildcard-words]]// parse into //[[CL:Glossary:strings]]//.
 
\endsubsubsubsection%{Wildcard Words in a Logical Pathname Namestring}

\beginsubsubsubsection{Lowercase Letters in a Logical Pathname Namestring}

When parsing //words// and //wildcard-words//,
lowercase letters are translated to uppercase.

\endsubsubsubsection%{Lowercase Letters in a Logical Pathname Namestring}

\beginsubsubsubsection{Other Syntax in a Logical Pathname Namestring}

The consequences of using characters other than those specified here
in a //[[CL:Glossary:logical pathname]]// //[[CL:Glossary:namestring]]// are unspecified.



The consequences of using any value not specified here as a 
//[[CL:Glossary:logical pathname]]// component are unspecified.

\endsubsubsubsection%{Other Syntax in a Logical Pathname Namestring}





\endsubsubsection%{Additional Information about Parsing Logical Pathname Namestrings}

\endSubsection%{Syntax of Logical Pathname Namestrings}

\beginSubsection{Logical Pathname Components}

\beginsubsubsection{Unspecific Components of a Logical Pathname}
\DefineSection{LogicalPathCompUnspecific}

The device component of a //[[CL:Glossary:logical pathname]]// is always **'':unspecific''**;
no other component of a \term {logical pathname} can be **'':unspecific''**.  

\endsubsubsection%{Unspecific Components of a Logical Pathname}

\beginsubsubsection{Null Strings as Components of a Logical Pathname}

The null string, \f{""}, is not a valid value for any component of a //[[CL:Glossary:logical pathname]]//.














\endsubsubsection%{Null Strings as Components of a Logical Pathname}

\endSubsection%{Logical Pathname Components}
