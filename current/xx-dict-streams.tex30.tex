====== Function OPEN ======

====Syntax====
\DefunWithValuesNewline open {filespec ''&key'' direction element-type if-exists if-does-not-exist external-format} {stream}

====Arguments and Values====
//filespec// - a //[[CL:Glossary:pathname designator]]//.

//direction// - one of **'':input''**, **'':output''**, **'':io''**, or **'':probe''**. The default is **'':input''**.

//element-type// - a //[[CL:Glossary:type specifier]]// for //[[CL:Glossary:recognizable subtype]]// of **[[CL:Types:character]]**; or a //[[CL:Glossary:type specifier]]// for a //[[CL:Glossary:finite]]// //[[CL:Glossary:recognizable subtype]]// of //[[CL:Glossary:integer]]//; or one of the //[[CL:Glossary:symbols]]// \misc{signed-byte}, \misc{unsigned-byte}, or **'':default''**. The default is **[[CL:Types:character]]**.

//if-exists// - one of **'':error''**, **'':new-version''**, **'':rename''**, **'':rename-and-delete''**, **'':overwrite''**, **'':append''**, **'':supersede''**, or **[[CL:Constant Variables:nil]]**. The default is **'':new-version''** if the version component of //filespec// is **'':newest''**, or **'':error''** otherwise.

//if-does-not-exist// - one of **'':error''**, **'':create''**, or **[[CL:Constant Variables:nil]]**. The default is **'':error''** if //direction// is **'':input''** or //if-exists// is **'':overwrite''** or **'':append''**; **'':create''** if //direction// is **'':output''** or **'':io''**, and //if-exists// is neither **'':overwrite''** nor **'':append''**; or **[[CL:Constant Variables:nil]]** when //direction// is **'':probe''**.

//external-format// - an //[[CL:Glossary:external file format designator]]//. The default is **'':default''**.

//stream// - a //[[CL:Glossary:file stream]]// or **[[CL:Constant Variables:nil]]**.

====Description====
**[[CL:Functions:open]]** creates, opens, and returns a //[[CL:Glossary:file stream]]// that is connected to the file specified by //filespec//. //Filespec// is the name of the file to be opened.

If the //filespec// //[[CL:Glossary:designator]]// is a //[[CL:Glossary:stream]]//, that //[[CL:Glossary:stream]]// is not closed first or otherwise affected.

The keyword arguments to **[[CL:Functions:open]]** specify the characteristics of the //[[CL:Glossary:file stream]]// that is returned, and how to handle errors.

If //direction// is **'':input''**

or **'':probe''**, or if //if-exists// is not **'':new-version''** and the version component of the //filespec// is **'':newest''**, then the file opened is that file already existing in the file system that has a version greater than that of any other file in the file system whose other pathname components are the same as those of //filespec//.

An implementation is required to recognize all of the **[[CL:Functions:open]]** keyword options and to do something reasonable in the context of the host operating system. For example, if a file system does not support distinct file versions and does not distinguish the notions of deletion and expunging, **'':new-version''** might be treated the same as **'':rename''** or **'':supersede''**, and **'':rename-and-delete''** might be treated the same as **'':supersede''**.

\beginlist

\itemitem{**'':direction''**}

These are the possible values for //direction//, and how they affect the nature of the //[[CL:Glossary:stream]]// that is created:

\beginlist

\itemitem{**'':input''**}

Causes the creation of an //[[CL:Glossary:input]]// //[[CL:Glossary:file stream]]//.

\itemitem{**'':output''**}

Causes the creation of an //[[CL:Glossary:output]]// //[[CL:Glossary:file stream]]//.

\itemitem{**'':io''**}

Causes the creation of a //[[CL:Glossary:bidirectional]]// //[[CL:Glossary:file stream]]//.

\itemitem{**'':probe''**}

Causes the creation of a "no-directional" //[[CL:Glossary:file stream]]//; in effect, the //[[CL:Glossary:file stream]]// is created and then closed prior to being returned by **[[CL:Functions:open]]**.

\endlist

\itemitem{**'':element-type''**}

The //element-type// specifies the unit of transaction for the //[[CL:Glossary:file stream]]//.

If it is **'':default''**, the unit is determined by //[[CL:Glossary:file system]]//, possibly based on the //[[CL:Glossary:file]]//.

\itemitem{**'':if-exists''**}

//if-exists// specifies the action to be taken if //direction// is **'':output''** or **'':io''** and a file of the name //filespec// already exists. If //direction// is **'':input''**, not supplied, or **'':probe''**, //if-exists// is ignored. These are the results of **[[CL:Functions:open]]** as modified by //if-exists//:

\beginlist

\itemitem{**'':error''**}

An error of type **[[CL:Types:file-error]]** is signaled.

\itemitem{**'':new-version''**}

A new file is created with a larger version number.

\itemitem{**'':rename''**}

The existing file is renamed to some other name and then a new file is created.

\itemitem{**'':rename-and-delete''**}

The existing file is renamed to some other name, then it is deleted but not expunged, and then a new file is created.

\itemitem{**'':overwrite''**}

Output operations on the //[[CL:Glossary:stream]]// destructively modify the existing file. If //direction// is **'':io''** the file is opened in a bidirectional mode that allows both reading and writing. The file pointer is initially positioned at the beginning of the file; however, the file is not truncated back to length zero when it is opened.

\itemitem{**'':append''**}

Output operations on the //[[CL:Glossary:stream]]// destructively modify the existing file. The file pointer is initially positioned at the end of the file.

If //direction// is **'':io''**, the file is opened in a bidirectional mode that allows both reading and writing.

\itemitem{**'':supersede''**}

The existing file is superseded; that is, a new file with the same name as the old one is created. If possible, the implementation should not destroy the old file until the new //[[CL:Glossary:stream]]// is closed.

\itemitem{**[[CL:Constant Variables:nil]]**}

No file or //[[CL:Glossary:stream]]// is created; instead, **[[CL:Constant Variables:nil]]** is returned to indicate failure.

\endlist

\itemitem{**'':if-does-not-exist''**}

//if-does-not-exist// specifies the action to be taken if a file of name //filespec// does not already exist. These are the results of **[[CL:Functions:open]]** as modified by //if-does-not-exist//:

\beginlist

\itemitem{**'':error''**}

An error of type **[[CL:Types:file-error]]** is signaled.

\itemitem{**'':create''**}

An empty file is created. Processing continues as if the file had already existed but no processing as directed by //if-exists// is performed.

\itemitem{**[[CL:Constant Variables:nil]]**}

No file or //[[CL:Glossary:stream]]// is created; instead, **[[CL:Constant Variables:nil]]** is returned to indicate failure.

\endlist

\itemitem{**'':external-format''**}

This option selects an //[[CL:Glossary:external file format]]// for the //[[CL:Glossary:file]]//:

The only //[[CL:Glossary:standardized]]// value for this option is **'':default''**, although //[[CL:Glossary:implementations]]// are permitted to define additional //[[CL:Glossary:external file formats]]// and //[[CL:Glossary:implementation-dependent]]// values returned by **[[CL:Functions:stream-external-format]]** can also be used by //[[CL:Glossary:conforming programs]]//.


The //external-format// is meaningful for any kind of //[[CL:Glossary:file stream]]// whose //[[CL:Glossary:element type]]// is a //[[CL:Glossary:subtype]]// of //[[CL:Glossary:character]]//. This option is ignored for //[[CL:Glossary:streams]]// for which it is not meaningful; however, //[[CL:Glossary:implementations]]// may define other //[[CL:Glossary:element types]]// for which it is meaningful.

The consequences are unspecified if a //[[CL:Glossary:character]]// is written that cannot be represented by the given //[[CL:Glossary:external file format]]//.

\endlist

When a file is opened, a //[[CL:Glossary:file stream]]// is constructed to serve as the file system's ambassador to the Lisp environment; operations on the //[[CL:Glossary:file stream]]// are reflected by operations on the file in the file system.

A file can be deleted, renamed, or destructively modified by **[[CL:Functions:open]]**.

For information about opening relative pathnames, see section {\secref\MergingPathnames}.

====Examples====
<blockquote> (open ''filespec'' :direction :probe) → #<Closed Probe File Stream...> ([[CL:Macros:defparameter]] q (merge-pathnames (user-homedir-pathname) "test")) → #<PATHNAME :HOST NIL :DEVICE ''device-name'' :DIRECTORY ''directory-name'' :NAME "test" :TYPE NIL :VERSION :NEWEST> (open ''filespec'' :if-does-not-exist :create) → #<Input File Stream...> ([[CL:Macros:defparameter]] s (open ''filespec'' :direction :probe)) → #<Closed Probe File Stream...> (truename s) → #<PATHNAME :HOST NIL :DEVICE ''device-name'' :DIRECTORY ''directory-name'' :NAME ''filespec'' :TYPE ''extension'' :VERSION 1> (open s :direction :output :if-exists nil) → NIL </blockquote>

====Affected By====
The nature and state of the host computer's //[[CL:Glossary:file system]]//.

====Exceptional Situations====
If //if-exists// is **'':error''**, (subject to the constraints on the meaning of //if-exists// listed above), an error of type **[[CL:Types:file-error]]** is signaled.

If //if-does-not-exist// is **'':error''** (subject to the constraints on the meaning of //if-does-not-exist// listed above), an error of type **[[CL:Types:file-error]]** is signaled.

If it is impossible for an implementation to handle some option in a manner close to what is specified here, an error of type **[[CL:Types:error]]** might be signaled.

An error of type **[[CL:Types:file-error]]** is signaled if **[[CL:Functions:(wild-pathname-p //filespec//)]]** returns true.

An error of type **[[CL:Types:error]]** is signaled if the //external-format// is not understood by the //[[CL:Glossary:implementation]]//.

The various //[[CL:Glossary:file systems]]// in existence today have widely differing capabilities, and some aspects of the //[[CL:Glossary:file system]]// are beyond the scope of this specification to define. A given //[[CL:Glossary:implementation]]// might not be able to support all of these options in exactly the manner stated. An //[[CL:Glossary:implementation]]// is required to recognize all of these option keywords and to try to do something "reasonable" in the context of the host //[[CL:Glossary:file system]]//. Where necessary to accomodate the //[[CL:Glossary:file system]]//, an //[[CL:Glossary:implementation]]// deviate slightly from the semantics specified here without being disqualified for consideration as a //[[CL:Glossary:conforming implementation]]//. If it is utterly impossible for an //[[CL:Glossary:implementation]]// to handle some option in a manner similar to what is specified here, it may simply signal an error.

With regard to the **'':element-type''** option, if a //[[CL:Glossary:type]]// is requested that is not supported by the //[[CL:Glossary:file system]]//, a substitution of types such as that which goes on in //[[CL:Glossary:upgrade|upgrading]]// is permissible. As a minimum requirement, it should be the case that opening an //[[CL:Glossary:output]]// //[[CL:Glossary:stream]]// to a //[[CL:Glossary:file]]// in a given //[[CL:Glossary:element type]]// and later opening an //[[CL:Glossary:input]]// //[[CL:Glossary:stream]]// to the same //[[CL:Glossary:file]]// in the same //[[CL:Glossary:element type]]// should work compatibly.

====See Also====
**[[CL:Functions:with-open-file]]**, **[[CL:Functions:close]]**, **[[CL:Types:pathname]]**, **[[CL:Types:logical-pathname]]**,{\secref\MergingPathnames},

{\secref\PathnamesAsFilenames}

====Notes====
**[[CL:Functions:open]]** does not automatically close the file when an abnormal exit occurs.

When //element-type// is a //[[CL:Glossary:subtype]]// of **[[CL:Types:character]]**, **[[CL:Functions:read-char]]** and/or **[[CL:Functions:write-char]]** can be used on the resulting //[[CL:Glossary:file stream]]//.

When //element-type// is a //[[CL:Glossary:subtype]]// of //[[CL:Glossary:integer]]//, **[[CL:Functions:read-byte]]** and/or **[[CL:Functions:write-byte]]** can be used on the resulting //[[CL:Glossary:file stream]]//.

When //element-type// is **'':default''**, the //[[CL:Glossary:type]]// can be determined by using **[[CL:Functions:stream-element-type]]**.

\issue{CHARACTER-PROPOSAL:2-5-2} \issue{STREAM-ACCESS:ADD-TYPES-ACCESSORS} \issue{CHARACTER-PROPOSAL:2-5-2} \issue{PATHNAME-WILD:NEW-FUNCTIONS} \issue{CHARACTER-PROPOSAL:2-5-2} \issue{PATHNAME-LOGICAL:ADD} \issue{FILE-OPEN-ERROR:SIGNAL-FILE-ERROR} \issue{PATHNAME-HOST-PARSING:RECOGNIZE-LOGICAL-HOST-NAMES}
