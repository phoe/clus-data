====== Special Operator UNWIND-PROTECT ======

====Syntax====
  * **unwind-protect** //protected-form cleanup-form''*''// → //result''*''//

====Arguments and Values====
  * //protected-form// - a //[[CL:Glossary:form]]//.
  * //cleanup-form// - a //[[CL:Glossary:form]]//.
  * //results// - the //[[CL:Glossary:value|values]]// of the //[[CL:Glossary:protected-form]]//.

====Description====
**unwind-protect** evaluates //protected-form// and guarantees that //cleanup-forms// are executed before **unwind-protect** exits, whether it terminates normally or is aborted by a control transfer of some kind. **unwind-protect** is intended to be used to make sure that certain side effects take place after the evaluation of //protected-form//.

If a //[[CL:Glossary:non-local exit]]// occurs during execution of //cleanup-forms//, no special action is taken. The //cleanup-forms// of **unwind-protect** are not protected by that **unwind-protect**.

**unwind-protect** protects against all attempts to exit from //protected-form//, including **[[CL:Special Operators:go]]**, **[[CL:Macros:handler-case]]**, **[[CL:Macros:ignore-errors]]**, **[[CL:Macros:restart-case]]**, **[[CL:Special Operators:return-from]]**, **[[CL:Special Operators:throw]]**, and **[[CL:Macros:with-simple-restart]]**.

Undoing of //[[CL:Glossary:handler]]// and //[[CL:Glossary:restart]]// //[[CL:Glossary:binding|bindings]]// during an exit happens in parallel with the undoing of the bindings of //[[CL:Glossary:dynamic variable|dynamic variables]]// and **[[CL:Special Operators:catch]]** tags, in the reverse order in which they were established. The effect of this is that //cleanup-form// sees the same //[[CL:Glossary:handler]]// and //[[CL:Glossary:restart]]// //[[CL:Glossary:binding|bindings]]//, as well as //[[CL:Glossary:dynamic variable]]// //[[CL:Glossary:binding|bindings]]// and **[[CL:Special Operators:catch]]** tags, as were visible when the **unwind-protect** was entered.

====Examples==== 
In the below example, when **[[CL:Special Operators:go]]** is executed, the call to **[[CL:Functions:print]]** is executed first, and then the transfer of control to the tag ''out'' is completed.

<blockquote>
([[CL:Special Operators:tagbody]] 
   ([[CL:Special Operators:let]] ((x 3)) 
     (unwind-protect 
         ([[CL:Macros:when]] ([[CL:Functions:numberp]] x) ([[CL:Special Operators:go]] out)) 
       ([[CL:Functions:print]] x))) 
 out 
   ...)
</blockquote>

-------

<blockquote>
([[CL:Macros:defvar]] *state*) <r>*STATE*</r>
([[CL:Macros:defun]] dummy-function (x) 
  ([[CL:Macros:setf]] *state* 'running) 
  ([[CL:Macros:unless]] ([[CL:Functions:numberp]] x) 
    ([[CL:Special Operators:throw]] 'abort 'not-a-number))
  ([[CL:Macros:setf]] *state* ([[CL:Functions:math-one-plus|1+]] x))) <r>DUMMY-FUNCTION </r>
([[CL:Special Operators:catch]] 'abort (dummy-function 1)) <r>2 </r>
*state* <r>2 </r>
([[CL:Special Operators:catch]] 'abort (dummy-function 'trash)) <r>NOT-A-NUMBER</r>
*state* <r>RUNNING </r>
([[CL:Special Operators:catch]] 'abort 
  (unwind-protect 
      (dummy-function 'trash) 
    ([[CL:Macros:setf]] *state* 'aborted))) <r>NOT-A-NUMBER </r>
*state* <r>ABORTED</r>
</blockquote>

The following code is not correct:

<blockquote>
(unwind-protect
    ([[CL:Special Operators:progn]
      ([[CL:Macros:incf]] *access-count*)
      (perform-access))
  ([[CL:Macros:decf]] *access-count*)) 
</blockquote>

If an exit occurs before completion of **[[CL:Macros:incf]]**, the **[[CL:Macros:decf]]** //[[CL:Glossary:form]]// is executed anyway, resulting in an incorrect value for ''*access-count*''. The correct way to code this is as follows:

<blockquote>
([[CL:Special Operators:let]] ((old-count *access-count*))
  (unwind-protect
    ([[CL:Special Operators:progn]]
      ([[CL:Macros:incf]] *access-count*)
      (perform-access)) 
    ([[CL:Macros:setf]] *access-count* old-count)))
</blockquote>

--------

<blockquote> 
([[CL:Special Operators:block]] [[CL:Constant Variables:nil]]
  (unwind-protect 
      ([[CL:Macros:return]] 1)
    ([[CL:Macros:return]] 2))) <r>2</r>

([[CL:Special Operators:block]] a 
  ([[CL:Special Operators:block]] b 
    (unwind-protect 
        ([[CL:Special Operators:return-from]] a 1) 
      ([[CL:Special Operators:return-from]] b 2))))
<u>This has undefined consequences.</u>

([[CL:Special Operators:catch]] [[CL:Constant Variables:nil]] 
  (unwind-protect 
      ([[CL:Special Operators:throw]] [[CL:Constant Variables:nil]] 1) 
    ([[CL:Special Operators:throw]] [[CL:Constant Variables:nil]] 2))) <r>2</r>
</blockquote>

The following has undefined consequences because the **[[CL:Special Operators:catch]]** of ''b'' is passed over by the first **[[CL:Special Operators:throw]]**, hence //[[CL:Glossary:portable program|portable programs]]// must assume its //[[CL:Glossary:dynamic extent]]// is terminated. The //[[CL:Glossary:binding]]// of the **[[CL:Special Operators:catch]]** tag is not yet disestablished and therefore it is the target of the second **[[CL:Special Operators:throw]]**. 

<blockquote> 
([[CL:Special Operators:catch]] 'a 
  ([[CL:Special Operators:catch]] 'b 
    (unwind-protect 
      ([[CL:Special Operators:throw]] 'a 1) 
    ([[CL:Special Operators:throw]] 'b 2))))
</blockquote>

<blockquote> 
([[CL:Special Operators:catch]] 'foo
  ([[CL:Functions:format]] [[CL:Constant Variables:t]] "The inner catch returns ~s.~%"
    ([[CL:Special Operators:catch]] 'foo 
      (unwind-protect 
          ([[CL:Special Operators:throw]] 'foo :first-throw) 
        ([[CL:Special Operators:throw]] 'foo :second-throw))))
  :outer-catch)
<o>The inner catch returns :SECOND-THROW</o>
<r>:OUTER-CATCH</r>
</blockquote>

In the following example, the inner **[[CL:Special Operators:catch]]** of ''a'' is passed over, but because that **[[CL:Special Operators:catch]]** is disestablished before the **[[CL:Special Operators:throw]]** to ''a'' is executed, it isn't seen. 

<blockquote>  
([[CL:Special Operators:catch]] 'a 
  ([[CL:Special Operators:catch]] 'b 
    (unwind-protect 
        ([[CL:Functions:math-one-plus|1+]] ([[CL:Special Operators:catch]] 'a ([[CL:Special Operators:throw]] 'b 1))) 
      ([[CL:Special Operators:throw]] 'a 10)))) <r>10</r>
</blockquote>

The following has undefined consequences because the extent of the ''(catch 'bar ...)'' exit ends when the ''(throw 'foo ...)'' commences. 

<blockquote> 
([[CL:Special Operators:catch]] 'foo 
  ([[CL:Special Operators:catch]] 'bar 
    (unwind-protect 
        ([[CL:Special Operators:throw]] 'foo 3) 
      ([[CL:Special Operators:throw]] 'bar 4) 
      ([[CL:Functions:print]] 'xxx))))
<u>This has undefined consequences.</u>
</blockquote>

In the following example, the ''(throw 'foo ...)'' has no effect on the scope of the ''bar'' catch tag or the extent of the ''(catch 'bar ...)'' exit. 

<blockquote> 
([[CL:Special Operators:catch]] 'bar 
  ([[CL:Special Operators:catch]] 'foo 
    (unwind-protect 
        ([[CL:Special Operators:throw]] 'foo 3) 
      ([[CL:Special Operators:throw]] 'bar 4) 
      ([[CL:Functions:print]] 'xxx))))
<r>4</r>
</blockquote>

<blockquote>
([[CL:Special Operators:block]] [[CL:Constant Variables:nil]] 
  ([[CL:Special Operators:let]] ((x 5))
    ([[CL:Symbols:declare]] ([[CL:Declarations:special]] x))
    (unwind-protect 
        ([[CL:Macros:return]]) 
      ([[CL:Functions:print]] x))))
<o>5</o>
<r>NIL</r>
</blockquote>

====Affected By====
None.

====Exceptional Situations====
None.

====See Also====
  * **[[CL:Special Operators:catch|Special Operator CATCH]]**
  * **[[CL:Special Operators:go|Special Operator GO]]**
  * **[[CL:Macros:handler-case|Macro HANDLER-CASE]]**
  * **[[CL:Macros:restart-case|Macro RESTART-CASE]]**
  * **[[CL:Macros:return|Macro RETURN]]**
  * **[[CL:Special Operators:return-from|Special Operator RETURN-FROM]]**
  * **[[CL:Special Operators:throw|Special Operator THROW]]**
  * {\secref\Evaluation}

====Notes====
None.

\issue{EXIT-EXTENT-AND-CONDITION-SYSTEM:LIKE-DYNAMIC-BINDINGS} \issue{EXIT-EXTENT:MINIMAL}
