====== Function PPRINT-TAB ======

====Syntax====
**pprint-tab {kind colnum colinc** //\opt} stream// → //**[[CL:Constant Variables:nil]]**//

====Arguments and Values====
//kind// - one of **'':line''**, **'':section''**, **'':line-relative''**, or **'':section-relative''**.

//colnum// - a non-negative //[[CL:Glossary:integer]]//.

//colinc// - a non-negative //[[CL:Glossary:integer]]//.

//stream// - an //[[CL:Glossary:output]]// //[[CL:Glossary:stream designator]]//.

====Description====
Specifies tabbing to //stream// as performed by the standard \formatdirective{T}.

If //stream// is a //[[CL:Glossary:pretty printing stream]]// and the //[[CL:Glossary:value]]// of **[[CL:Variables:star-print-pretty-star|*print-pretty*]]** is //[[CL:Glossary:true]]//,

tabbing is performed; otherwise, **[[CL:Functions:pprint-tab]]** has no effect.

The arguments //colnum// and //colinc// correspond to the two //parameters// to \formatOp{T} and are in terms of //[[CL:Glossary:ems]]//. The //kind// argument specifies the style of tabbing. It must be one of **'':line''** (tab as by \formatOp{T}), **'':section''** (tab as by \formatOp{:T}, but measuring horizontal positions relative to the start of the dynamically enclosing section), **'':line-relative''** (tab as by \formatOp{@T}), or **'':section-relative''** (tab as by \formatOp{:@T}, but measuring horizontal positions relative to the start of the dynamically enclosing section).

====Examples====
None.

====Side Effects====
None.

====Affected By====
None.

====Exceptional Situations====
An error is signaled if //kind// is not one of **'':line''**, **'':section''**, **'':line-relative''**, or **'':section-relative''**.

====See Also====
**[[CL:Macros:pprint-logical-block]]**

====Notes====
None.

\issue{PRETTY-PRINT-INTERFACE} \issue{GENERALIZE-PRETTY-PRINTER:UNIFY}
