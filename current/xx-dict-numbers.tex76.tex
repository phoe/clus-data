\begincom{most-positive-short-float, least-positive-short-float, least-positive-normalized-short-float, most-positive-double-float, least-positive-double-float, least-positive-normalized-double-float, most-positive-long-float, least-positive-long-float, least-positive-normalized-long-float, most-positive-single-float, least-positive-single-float, least-positive-normalized-single-float, most-negative-short-float, least-negative-short-float, least-negative-normalized-short-float, most-negative-single-float, least-negative-single-float, least-negative-normalized-single-float, most-negative-double-float, least-negative-double-float, least-negative-normalized-double-float, most-negative-long-float, least-negative-long-float, least-negative-normalized-long-float}\ftype{Constant Variable}

====Constant Value====

//[[CL:Glossary:implementation-dependent]]//.

====Description====

These //[[CL:Glossary:constant variables]]// provide a way for programs to examine the //[[CL:Glossary:implementation-defined]]// limits for the various float formats.

Of these //[[CL:Glossary:variables]]//, each which has ""-normalized'''' in its //[[CL:Glossary:name]]// must have a //[[CL:Glossary:value]]// which is a //[[CL:Glossary:normalized]]// //[[CL:Glossary:float]]//, and each which does not have ""-normalized'''' in its name may have a //[[CL:Glossary:value]]// which is either a //[[CL:Glossary:normalized]]// //[[CL:Glossary:float]]// or a //[[CL:Glossary:denormalized]]// //[[CL:Glossary:float]]//, as appropriate.

Of these //[[CL:Glossary:variables]]//, each which has ""short-float'''' in its name must have a //[[CL:Glossary:value]]// which is a //[[CL:Glossary:short float]]//, each which has ""single-float'''' in its name must have a //[[CL:Glossary:value]]// which is a //[[CL:Glossary:single float]]//, each which has ""double-float'''' in its name must have a //[[CL:Glossary:value]]// which is a //[[CL:Glossary:double float]]//, and each which has ""long-float'''' in its name must have a //[[CL:Glossary:value]]// which is a //[[CL:Glossary:long float]]//.

\beginlist

\itemitem{\bull} **[[CL:Constant Variables:most-positive-short-float]]**, **[[CL:Constant Variables:most-positive-single-float]]**, **[[CL:Constant Variables:most-positive-double-float]]**, **[[CL:Constant Variables:most-positive-long-float]]** \Vskip 18pt!

Each of these //[[CL:Glossary:constant variables]]// has as its //[[CL:Glossary:value]]// the positive //[[CL:Glossary:float]]// of the largest magnitude (closest in value to, but not equal to, positive infinity) for the float format implied by its name.

\itemitem{\bull} **[[CL:Constant Variables:least-positive-short-float]]**, **[[CL:Constant Variables:least-positive-normalized-short-float]]**, **[[CL:Constant Variables:least-positive-single-float]]**, **[[CL:Constant Variables:least-positive-normalized-single-float]]**, **[[CL:Constant Variables:least-positive-double-float]]**, **[[CL:Constant Variables:least-positive-normalized-double-float]]**, **[[CL:Constant Variables:least-positive-long-float]]**, **[[CL:Constant Variables:least-positive-normalized-long-float]]** \Vskip 42pt!

Each of these //[[CL:Glossary:constant variables]]// has as its //[[CL:Glossary:value]]// the smallest positive (nonzero) //[[CL:Glossary:float]]// for the float format implied by its name.

\itemitem{\bull} **[[CL:Constant Variables:least-negative-short-float]]**, **[[CL:Constant Variables:least-negative-normalized-short-float]]**, **[[CL:Constant Variables:least-negative-single-float]]**, **[[CL:Constant Variables:least-negative-normalized-single-float]]**, **[[CL:Constant Variables:least-negative-double-float]]**, **[[CL:Constant Variables:least-negative-normalized-double-float]]**, **[[CL:Constant Variables:least-negative-long-float]]**, **[[CL:Constant Variables:least-negative-normalized-long-float]]** \Vskip 42pt!

Each of these //[[CL:Glossary:constant variables]]// has as its //[[CL:Glossary:value]]// the negative (nonzero) //[[CL:Glossary:float]]// of the smallest magnitude for the float format implied by its name. (If an implementation supports minus zero as a //[[CL:Glossary:different]]// //[[CL:Glossary:object]]// from positive zero, this value must not be minus zero.)

\itemitem{\bull} **[[CL:Constant Variables:most-negative-short-float]]**, **[[CL:Constant Variables:most-negative-single-float]]**, **[[CL:Constant Variables:most-negative-double-float]]**, **[[CL:Constant Variables:most-negative-long-float]]** \Vskip 18pt!

Each of these //[[CL:Glossary:constant variables]]// has as its //[[CL:Glossary:value]]// the negative //[[CL:Glossary:float]]// of the largest magnitude (closest in value to, but not equal to, negative infinity) for the float format implied by its name.

\endlist

====Examples====

None.

====See Also====

None.

====Notes====

\issue{FLOAT-UNDERFLOW:ADD-VARIABLES}
