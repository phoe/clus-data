====== Function PACKAGE-NAME ======

====Syntax====
**package-name** //package// → //name//

====Arguments and Values====
//package// - a //[[CL:Glossary:package designator]]//.

//name// - a //[[CL:Glossary:string]]//

or **[[CL:Constant Variables:nil]]**.

====Description====
**[[CL:Functions:package-name]]** returns the //[[CL:Glossary:string]]// that names //package//,

or **[[CL:Constant Variables:nil]]** if the //package// //[[CL:Glossary:designator]]// is a //[[CL:Glossary:package]]// //[[CL:Glossary:object]]// that has no name (see the //[[CL:Glossary:function]]// **[[CL:Functions:delete-package]]**).

====Examples====
<blockquote> (in-package "COMMON-LISP-USER") → #<PACKAGE "COMMON-LISP-USER"> (package-name *package*) → "COMMON-LISP-USER" (package-name (symbol-package :test)) → "KEYWORD" (package-name (find-package 'common-lisp)) → "COMMON-LISP" </blockquote>

<blockquote> (defvar *foo-package* (make-package "FOO")) (rename-package "FOO" "FOO0") (package-name *foo-package*) → "FOO0" </blockquote>

====Side Effects====
None.

====Affected By====
None.

====Exceptional Situations====
Should signal an error of type type-error if //package// is not a //[[CL:Glossary:package designator]]//.

====See Also====
None.

====Notes====
None.

\issue{PACKAGE-FUNCTION-CONSISTENCY:MORE-PERMISSIVE} \issue{PACKAGE-DELETION:NEW-FUNCTION} \issue{PACKAGE-DELETION:NEW-FUNCTION} \issue{PACKAGE-FUNCTION-CONSISTENCY:MORE-PERMISSIVE}
