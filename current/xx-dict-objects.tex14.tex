====== Function SLOT-VALUE ======

====Syntax====
  * **slot-value** //object slot-name// → //value//

====Arguments and Values====
  * //object// - an //[[CL:Glossary:object]]//.
  * //name// - a //[[CL:Glossary:symbol]]//.
  * //value// - an //[[CL:Glossary:object]]//.

====Description====
The //[[CL:Glossary:function]]// **slot-value** returns the //[[CL:Glossary:value]]// of the //[[CL:Glossary:slot]]// named //slot-name// in the //object//. If there is no //[[CL:Glossary:slot]]// named //slot-name//, **[[CL:Functions:slot-missing]]** is called. If the //[[CL:Glossary:slot]]// is unbound, **[[CL:Functions:slot-unbound]]** is called.

The macro **[[CL:Macros:setf]]** can be used with **slot-value** to change the value of a //[[CL:Glossary:slot]]//.

====Examples====
<blockquote>
([[CL:Macros:defclass]] foo () 
  ((a :accessor foo-a :initarg :a :initform 1) 
   (b :accessor foo-b :initarg :b) 
   (c :accessor foo-c :initform 3))) <r>#<STANDARD-CLASS FOO 244020371> </r>
([[CL:Macros:defparameter]] *foo-1* ([[CL:Functions:make-instance]] 'foo :a 'one :b 'two)) <r>#<FOO 36325624> </r>
([[CL:Functions:slot-value]] *foo-1* 'a) <r>ONE </r>
([[CL:Functions:slot-value]] *foo-1* 'b) <r>TWO </r>
([[CL:Functions:slot-value]] *foo-1* 'c) <r>3 </r>
([[CL:Macros:setf]] (slot-value *foo-1* 'a) 'uno) <r>UNO </r>
([[CL:Functions:slot-value]] *foo-1* 'a) <r>UNO </r>
([[CL:Macros:defmethod]] foo-method ((x foo)) 
  ([[CL:Functions:slot-value]] x 'a)) <r>#<STANDARD-METHOD FOO-METHOD (FOO) 42720573> </r>
(foo-method *foo-1*) <r>UNO</r>
</blockquote>

====Affected By====
None.

====Exceptional Situations====
If an attempt is made to read a //[[CL:Glossary:slot]]// and no //[[CL:Glossary:slot]]// of the name //slot-name// exists in the //object//, **[[CL:Functions:slot-missing]]** is called as follows:

<blockquote>
(slot-missing (class-of //instance//) //instance// //slot-name// 'slot-value)
</blockquote>

(If **[[CL:Functions:slot-missing]]** is invoked, its //[[CL:Glossary:primary value]]// is returned by **slot-value**.)

----

If an attempt is made to write a //[[CL:Glossary:slot]]// and no //[[CL:Glossary:slot]]// of the name //slot-name// exists in the //object//, **[[CL:Functions:slot-missing]]** is called as follows:

<blockquote>
(slot-missing (class-of //instance//) //instance// //slot-name// 'setf //new-value//)
</blockquote>

(If **[[CL:Functions:slot-missing]]** returns in this case, any //[[CL:Glossary:values]]// are ignored.)

The specific behavior depends on //object//'s //[[CL:Glossary:metaclass]]//. An error is never signaled if //object// has //[[CL:Glossary:metaclass]]// **[[CL:Types:standard-class]]**. An error is always signaled if //object// has //[[CL:Glossary:metaclass]]// **[[CL:Types:built-in-class]]**. The consequences are unspecified if //object// has any other //[[CL:Glossary:metaclass]]// - an error might or might not be signaled in this situation. Note in particular that the behavior for //[[CL:Glossary:conditions]]// and //[[CL:Glossary:structures]]// is not specified.

====See Also====
  * **[[CL:Functions:slot-missing|Generic Function SLOT-MISSING]]**
  * **[[CL:Functions:slot-unbound|Generic Function SLOT-UNBOUND]]**
  * **[[CL:Macros:with-slots|Macro WITH-SLOTS]]**

====Notes====
Although no //[[CL:Glossary:implementation]]// is required to do so, implementors are strongly encouraged to implement the //[[CL:Glossary:function]]// **slot-value** using the //[[CL:Glossary:function]]// **[[CL:Functions:slot-value-using-class]]** described in the //[[CL:Glossary:Metaobject Protocol]]//. 

Implementations may optimize **slot-value** by compiling it inline.

\issue{SLOT-MISSING-VALUES:SPECIFY} \issue{SLOT-MISSING-VALUES:SPECIFY} \issue{SLOT-VALUE-METACLASSES:LESS-MINIMAL}
