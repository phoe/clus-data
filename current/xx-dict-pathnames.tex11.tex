\begincom{namestring, file-namestring, directory-namestring, host-namestring, enough-namestring}\ftype{Function}

====Syntax====

**namestring** //pathname// → //namestring//

**file-namestring ** //pathname// → //namestring// **directory-namestring** //pathname// → //namestring// **host-namestring ** //pathname// → //namestring//

**enough-namestring {pathname** //\opt} defaults// → //namestring//

====Arguments and Values====

//pathname// - a //[[CL:Glossary:pathname designator]]//.


//defaults// - a //[[CL:Glossary:pathname designator]]//.

\Default{\thevalueof{*default-pathname-defaults*}}

//namestring// - a //[[CL:Glossary:string]]// or **[[CL:Constant Variables:nil]]**. \editornote{KMP: Under what circumstances can NIL be returned??}

====Description====

These functions convert //pathname// into a namestring.

The name represented by //pathname// is returned as a //[[CL:Glossary:namestring]]// in an //[[CL:Glossary:implementation-dependent]]// canonical form.

**[[CL:Functions:namestring]]** returns the full form of //pathname//.

**[[CL:Functions:file-namestring]]** returns just the name, type, and version components of //pathname//.

**[[CL:Functions:directory-namestring]]** returns the directory name portion.

**[[CL:Functions:host-namestring]]** returns the host name.

**[[CL:Functions:enough-namestring]]** returns an abbreviated namestring that is just sufficient to identify the file named by //pathname// when considered relative to the //defaults//. It is required that

<blockquote> (merge-pathnames (enough-namestring pathname defaults) defaults) ≡ (merge-pathnames (parse-namestring pathname nil defaults) defaults) </blockquote> in all cases, and the result of **[[CL:Functions:enough-namestring]]** is the shortest reasonable //[[CL:Glossary:string]]// that will satisfy this criterion.

It is not necessarily possible to construct a valid //[[CL:Glossary:namestring]]// by concatenating some of the three shorter //[[CL:Glossary:namestrings]]// in some order.

====Examples====

<blockquote> (namestring "getty") → "getty" ([[CL:Macros:defparameter]] q (make-pathname :host "kathy" :directory (pathname-directory *default-pathname-defaults*) :name "getty")) → #S(PATHNAME :HOST "kathy" :DEVICE NIL :DIRECTORY ''directory-name'' :NAME "getty" :TYPE NIL :VERSION NIL) (file-namestring q) → "getty" (directory-namestring q) → ''directory-name'' (host-namestring q) → "kathy" </blockquote>

<blockquote> ;;;Using Unix syntax and the wildcard conventions used by the ;;;particular version of Unix on which this example was created: (namestring (translate-pathname "/usr/dmr/hacks/frob.l" "/usr/d*/hacks/*.l" "/usr/d*/backup/hacks/backup-*.*")) → "/usr/dmr/backup/hacks/backup-frob.l" (namestring (translate-pathname "/usr/dmr/hacks/frob.l" "/usr/d*/hacks/fr*.l" "/usr/d*/backup/hacks/backup-*.*")) → "/usr/dmr/backup/hacks/backup-ob.l"

;;;This is similar to the above example but uses two different hosts, ;;;U: which is a Unix and V: which is a VMS. Note the translation ;;;of file type and alphabetic case conventions. (namestring (translate-pathname "U:/usr/dmr/hacks/frob.l" "U:/usr/d*/hacks/*.l" "V:SYS''DISK:[D*.BACKUP.HACKS]BACKUP-*.*")) → "V:SYS''DISK:[DMR.BACKUP.HACKS]BACKUP-FROB.LSP" (namestring (translate-pathname "U:/usr/dmr/hacks/frob.l" "U:/usr/d*/hacks/fr*.l" "V:SYS''DISK:[D*.BACKUP.HACKS]BACKUP-*.*")) → "V:SYS''DISK:[DMR.BACKUP.HACKS]BACKUP-OB.LSP" </blockquote>


====Affected By====

None.

====Exceptional Situations====

None.

====See Also====

**[[CL:Functions:truename]]**, **[[CL:Functions:merge-pathnames]]**, **[[CL:Types:pathname]]**, **[[CL:Types:logical-pathname]]**,{\secref\FileSystemConcepts},

{\secref\PathnamesAsFilenames}

====Notes====

None.


\issue{PATHNAME-STREAM} \issue{PATHNAME-SYMBOL} \issue{PATHNAME-SYMBOL} \issue{PATHNAME-WILD:NEW-FUNCTIONS} \issue{PATHNAME-LOGICAL:ADD} \issue{FILE-OPEN-ERROR:SIGNAL-FILE-ERROR}
