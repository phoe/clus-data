\section{Conclusions and future work}

\subsection{Benefits and Disadvantages}

The benefits of my approach come as logical continuations of the slogans used in section \ref{requirements}.

The most obvious one, which is also the goal of the project, is the construction of a contemporary source of \cl{} documentation and a single resource capable of containing most of the knowledge a \cl{} programmer might need, under a license permitting its free modification, expansion and reuse.

Another upside is modernization of the specification by fixing its issues and bugs, expanding its examples sections, clarifying any inconsistencies and questions that have emerged since the creation of the standard and giving it a more aesthetically pleasing look. I consider it important for Lisp to have a modern, browsable documentation, especially in light of various comments that I have gathered throughout the Internet, which emphasize such a need.

A beneficial side effect of my approach is the generation of a version of the \cl{} specification in a markup format. Such a format can then be easily parsed by automated tools to produce a document of any required typesetting qualities.

\noindent\rule{\linewidth}{0.5pt}

The disadvantages of my current approach occur on different layers.

First of all, it is easy to keep a single static website on the Web for years without any changes, but CLUS is far from static because of its design. The body of code that CLUS will turn into, as the time progresses, will require maintenance in order to remain clear and readable; it will require reviewers to check the input from anyone wanting to contribute to the CLUS repositories.

Second, although it does apply specifically to the dpANS sources, parsing and hyperlinking the chapters of the specification takes significant time. Additionally, because of the variety of forms other bodies of Lisp documentation have, it will be non-trivial to import them into CLUS. It will require a separate effort to have them parsed and prepared for inclusion.

Third, the legal status and licensing issues of the various pieces of documentation will require separate thought. Creating a compilation work of all these elements will be essentially creating a derivative of them all and legal caution will need to be taken in case of documents with unknown or confusing legal status. It might be required to negotiate the terms of inclusion of particular pieces of work into CLUS with the respective holders of rights to them.

\subsection{Thoughts}

\subsubsection{dpANS as humanistic material}
Among all the literature available for studying \cl{}, I would like to mention the dpANS source files as a valuable reading matreial from a non-technical point of view.

The standard was created before the era of ubiquitous versioning systems. Because of this, the draft source contains many comments, some of them timestamped. They show the technical problems and decisions the langauge specifiers faced and solved in the process of creating a formal standard for a programming language. They also outline the features which were deprecated and removed---or, on the contrary, created and added along the way, some of which I personally find quite enlightening. What I want to emphasize here, though, is that they show X3J13 as a group of human beings working on a common goal. The comments there show various aspects of their work: from communicating messages between particular people, through decision-making and commented-out pieces of specification itself, to the in-jokes and humor of the people\footnote{Such as: \texttt{\%This list conjured by KMP, Barrett, and Loosemore. -kmp 14-Feb-92}}.

In my opinion, studying the original sources for all three draft previews (all of which are available online) might be valuable for any person who wants to research specification development or software development in general from a more human point of view as well as Lisp programmers who are interested in extending their background and the process through which \cl{} came to life.

\subsubsection{Translator or editor?}
Another thought that I would like to mention here is the fact that, in the beginning, I had imagined my work as simple translation of the sources from their \TeX{} format into wiki markup in order to let the DokuWiki engine format them into HTML. Reality has superseded these ideas---I quickly realized that the standard itself has its share of inconsistencies, bugs and other issues. It is of course expected for such a huge body of documentation to have issues and these issues do not undermine the value of the specification as a whole, but I have unexpectedly found myself to be able to fix them as I progress through the sources.

Suddenly, from a simple translator, I had become an editor of the \cl{} standard itself. What I am creating right now is not the draft sources being translated into DokuWiki markup---it is an edited version which contains many improvements and fixes to many issues that were impossible to fix in the previous CL specifications based on the work of X3J13.

It is a very responsible role that has emerged---but also one that I consider very satisfying.

\subsection{Mistakes}

I have made a mistake that is important enough for me to want to point it out here. Here is where my lack of experience as an editor of formal documents shows---I have not done any formal editing work before. When I began parsing the specification, I decided to fix all obvious errors and mistakes on the fly, as I progress through the text. Because of this, I have not separated the stages of converting the specification to the new markup and editing its text; changes in the text appear because one person---me, as an editor---decides that they should be changed.

This is contrary to the experience of Kent M. Pitman---\textit{``the primary qualification for Editorship was trustworthi\-ness---particularly, the ability to resist the urge to `meddle' in technical matters while acting in the role of `neutral editor'''}.\cite{kmp:2012:untold} Additionally, KMP noted that \textit{``it was necessary that the community have complete trust that the only reason a change might be made to the meaning of the language was if there was a corresponding technical change voted by the committee''}.\cite{kmp:2012:untold}

My situation is different from Kent's. The X3J13 had finished its work; the specification is done; there is no formal committee that an editor might need to obey or serve.  I do not (yet) work on editing the specification with other people directly. Nonetheless, the Lisp community is whom that I attempt to serve here, trying to create a basis for an extensible Lisp documentation that might in turn accomodate various needs of the community itself. It is a responsible role that I do not want to do loosely.

As I progressed through the text of the specification, I have made changes to its formal and informal parts as I saw fit and proper, asking the community every time I had any doubt about the original meaning or wanted to ask for support for a particular change. While my intention was to produce text that is clearer and better, it is obvious that such a frivolous and undocumented process applied to a strictly formal document produces text that is perhaps a better documentation - but I, from my own point of view, cannot call it a formal specification anymore.

This mistake that I have made has generated more work for me to do later. Once I finish converting the whole specification, I will need to make a review of the whole CLUS with the standard in one hand and the \us{} text in the other, pointing out the differences and collecting them out into a separate document listing the changes and differences between the two texts, so that such a list may later be reviewed by the Lisp community and the purpose of each change may be discussed. Only after such a document exists, I will consider the specification part of the \us{} to be formally complete.

\subsection{Plans}

It is impossible to speak of future plans without mentioning the Lisp community here.

The \cl{} \us{} was meant from the start to be a community-based project, meaning that it belongs to the Lisp community and is meant to be utilized and expanded within it. I hope that other people will aid me in my process by suggesting changes, submitting patches, possibly integrating the documentation for respective \cl{} libraries into the code and maintaining them later on.

There is an interesting project to convert the markup I have used into S-expressions\footnote{\url{https://www.reddit.com/r/lisp/comments/5xcafp/c/deje13k/}} that can then be parsed and understood by Lisp---as data---and text editors---as editable, formatted text. I hope that it will make it possible one day to freely edit the CLUS source code in ``lispy'' editors, such as Emacs or Climacs\footnote{\url{https://github.com/robert-strandh/Second-Climacs}}.

Once the specification is completely integrated, I intend to extend its scope to include common facilities and extensions included and/or used in most contemporary \cl{} implementations, such as the Metaobject Protocol, ASDF, \ql{} and the compatibility libraries which provide cross-platform functionalities not included in the standard such as concurrency or networking.

There are collections of issues regarding the specifications, found throughout the years by the community. Some are organized, such as the collection on CLiki\footnote{\url{http://www.cliki.net/proposed\%20ansi\%20revisions\%20and\%20clarifications}}; some, I hope, are in on the hard drives of the Lisp community. Previously, it was impossible to integrate them into proprietary texts of former editions of the specification, but with the creation of CLUS, this will no longer be the case.

I want to create quality standards for the respective types of pages and enforce them, in order to keep the quality of the documentation high and its style consistent across pages and modules.

In the far future, it might be possible that the CLUS might become a basis for a revised Common Lisp specification that might include changes like the CDRs\footnote{\url{https://common-lisp.net/project/cdr/}} or the CL21 project\footnote{\url{http://cl21.org/}}. I do not directly plan this far ahead, but I keep such a possibility in mind; I consider the kind of organic work that I am doing a required basis for any further attempts to improve the specification of \cl{}.

\subsection{Acknowledgements}

I would like to thank Professor Robert Strandh, doctors Włodzimierz and Małgorzata Moczurad, and Philipp Marek for general support during the creation of this work, including, but not limited to, proofreading, advice, and moral support; one more thank you to Robert Strandh for allowing me to use his \TeX{} layout for papers.

I would additionally like to thank the people from the \texttt{\#lisp} and \texttt{\#lisp-pl} IRC channel for a lot of support on this paper and the \cl{} \us{} project in general.
