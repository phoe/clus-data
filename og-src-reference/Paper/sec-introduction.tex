\section{Introduction}

The current state of \cl{} documentation in general is not satisfactory to me and many other people I have talked with throughout my relatively short experience as a Lisp programmer.

The bodies of documentation are often incomplete and out-of-sync---they do not reflect the current state of a particular library. They tend to be outdated and in need of modernization. They contain errors and mistakes and are either unmaintained or non-editable. They are non-hyperlinked and do not refer to each other in a way that makes it easy to navigate between them. They are scattered across the web and depend on multiple separated web hostings, increasing the risk of losing access to them if a particular server containing their data fails.

I see multiple issues this state of matters causes to \cl{} programmers, newbies and seasoned Lispers alike. The experienced programmers are most likely used to these pieces of documentation and that fact might minimize the losses they suffer, but, from my personal experience, newcomers are often lost and disoriented because of general fragmentation of the documentation, lack of maintenance, poor visual appeal, and lack of an obvious way to fix this state of things for those willing to do so.

For the sake of justice, I must remark that the things I have mentioned above do not disqualify said documentation from being useful and invaluable for many Common Lisp programmers. The manuals specifying the language, its extensions and many libraries are good enough as a building material and the multitude of Common Lisp applications proves this point. But I consider this state of matters to be improvable.

This paper contains an idea and requirements for an improvement of the state of Common Lisp documentation. It also contains a description of an implementation of this idea, the sources I have used, the full technological stack I have utilized so far, the problems and issues I have encountered, the benefits and disadvantages of my approach, a mention of a notable mistake I have made and the plans I have for the future extension and expansion of the aforementioned work.
