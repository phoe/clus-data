

A //[[CL:Glossary:type]]// is a (possibly infinite) set of //[[CL:Glossary:objects]]//. An //[[CL:Glossary:object]]// can belong to more than one //[[CL:Glossary:type]]//.   //[[CL:Glossary:Types]]// are never explicitly represented as //[[CL:Glossary:objects]]// by \clisp. Instead, they are referred to indirectly by the use of //[[CL:Glossary:type specifiers]]//, which are //[[CL:Glossary:objects]]// that denote //[[CL:Glossary:types]]//.

New //[[CL:Glossary:types]]// can be defined using \macref{deftype}, \macref{defstruct},  \macref{defclass}, and \macref{define-condition}.

\Thefunction{typep}, a set membership test, is used to determine whether a given //[[CL:Glossary:object]]// is of a given //[[CL:Glossary:type]]//.  The function **[[CL:Functions:subtypep]]**, a subset test, is used to determine whether a given //[[CL:Glossary:type]]// is a //[[CL:Glossary:subtype]]// of another given //[[CL:Glossary:type]]//.  The function **[[CL:Functions:type-of]]** returns a particular //[[CL:Glossary:type]]// to which a given //[[CL:Glossary:object]]// belongs, even though that //[[CL:Glossary:object]]// must belong to one or more other //[[CL:Glossary:types]]// as well. (For example, every //[[CL:Glossary:object]]// is \oftype{t}, 
 but **[[CL:Functions:type-of]]** always returns a //[[CL:Glossary:type specifier]]//
 for a //[[CL:Glossary:type]]// more specific than \typeref{t}.)

//[[CL:Glossary:Objects]]//, not //[[CL:Glossary:variables]]//, have //[[CL:Glossary:types]]//. Normally, any //[[CL:Glossary:variable]]// can have any //[[CL:Glossary:object]]// as its //[[CL:Glossary:value]]//. It is possible to declare that a //[[CL:Glossary:variable]]// takes on only  values of a given //[[CL:Glossary:type]]// by making an explicit //[[CL:Glossary:type declaration]]//.

//[[CL:Glossary:Types]]// are arranged in a directed acyclic graph, except for the presence of equivalences. 

//[[CL:Glossary:Declarations]]// can be made about //[[CL:Glossary:types]]// using \misc{declare},  **[[CL:Functions:proclaim]]**, \macref{declaim}, or \specref{the}. For more information about //[[CL:Glossary:declarations]]//, \seesection\Declarations.

Among the fundamental //[[CL:Glossary:objects]]// of the \CLOS\ are //[[CL:Glossary:classes]]//. A //[[CL:Glossary:class]]// determines the structure and behavior of a set of other //[[CL:Glossary:objects]]//, which are called its //[[CL:Glossary:instances]]//.  Every //[[CL:Glossary:object]]// is a //[[CL:Glossary:direct instance]]// of a //[[CL:Glossary:class]]//. The //[[CL:Glossary:class]]// of an //[[CL:Glossary:object]]// determines the set of operations that can be performed on the //[[CL:Glossary:object]]//. For more information, \seesection\Classes.

It is possible to write //[[CL:Glossary:functions]]// that have behavior //[[CL:Glossary:specialized]]// to the class of the //[[CL:Glossary:objects]]// which are their //[[CL:Glossary:arguments]]//. For more information, \seesection\GFsAndMethods.

The //[[CL:Glossary:class]]// of the //[[CL:Glossary:class]]// of an //[[CL:Glossary:object]]//  is called its //[[CL:Glossary:metaclass]]//. For more information about //[[CL:Glossary:metaclasses]]//, \seesection\MetaObjects.
