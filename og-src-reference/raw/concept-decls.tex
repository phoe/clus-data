

\newtermidx{Declarations}{declaration} provide a way of specifying information for use by program processors, such as the evaluator or the compiler.

\newtermidx{Local declarations}{local declaration} can be embedded in executable code using \misc{declare}. \newtermidx{Global declarations}{global declaration},  or \newtermidx{proclamations}{proclamation}, are established by **[[CL:Functions:proclaim]]** or \macref{declaim}.

\Thespecform{the} provides a shorthand notation for  making a //[[CL:Glossary:local declaration]]// about the //[[CL:Glossary:type]]// of the //[[CL:Glossary:value]]// of a given //[[CL:Glossary:form]]//.

The consequences are undefined if a program violates a //[[CL:Glossary:declaration]]// or a //[[CL:Glossary:proclamation]]//.

\beginSubsection{Minimal Declaration Processing Requirements}

In general, an //[[CL:Glossary:implementation]]// is free to ignore //[[CL:Glossary:declaration specifiers]]// except for the
     **[[CL:Declarations:declaration]]**\idxref{declaration},
     **[[CL:Declarations:notinline]]**\idxref{notinline},
     **[[CL:Declarations:safety]]**\idxref{safety},
 and **[[CL:Declarations:special]]**\idxref{special} //[[CL:Glossary:declaration specifiers]]//.

A **[[CL:Declarations:declaration]]** //[[CL:Glossary:declaration]]// must suppress warnings about unrecognized //[[CL:Glossary:declarations]]// of the kind that it declares. If an //[[CL:Glossary:implementation]]// does not produce warnings about unrecognized declarations, it may safely ignore this //[[CL:Glossary:declaration]]//.

A **[[CL:Declarations:notinline]]** //[[CL:Glossary:declaration]]// must be recognized by any //[[CL:Glossary:implementation]]// that supports inline functions or //[[CL:Glossary:compiler macros]]// in order to disable those facilities. An //[[CL:Glossary:implementation]]// that does not use inline functions or //[[CL:Glossary:compiler macros]]// may safely ignore this //[[CL:Glossary:declaration]]//.

A **[[CL:Declarations:safety]]** //[[CL:Glossary:declaration]]// that increases the current safety level  must always be recognized.  An //[[CL:Glossary:implementation]]// that always processes  code as if safety were high may safely ignore this //[[CL:Glossary:declaration]]//.

A **[[CL:Declarations:special]]** //[[CL:Glossary:declaration]]// must be processed by all //[[CL:Glossary:implementations]]//.

\endSubsection%{Minimal Declaration Processing Requirements}

\beginSubsection{Declaration Specifiers}

A //[[CL:Glossary:declaration specifier]]// is an //[[CL:Glossary:expression]]// that can appear at top level of a \misc{declare} expression or a \macref{declaim} form, or as  the argument to **[[CL:Functions:proclaim]]**. It is a //[[CL:Glossary:list]]// whose //[[CL:Glossary:car]]// is a //[[CL:Glossary:declaration identifier]]//, and whose //[[CL:Glossary:cdr]]// is data interpreted according to rules specific to the //[[CL:Glossary:declaration identifier]]//.

\endSubsection%{Declaration Specifiers}

\beginSubsection{Declaration Identifiers}

\Thenextfigure\ shows a list of all  //[[CL:Glossary:declaration identifiers]]//\idxterm{declaration identifier}  defined by this standard.

\issue{DECLARE-FUNCTION-AMBIGUITY:DELETE-FTYPE-ABBREVIATION} \displaythree{Common Lisp Declaration Identifiers}{ declaration&ignore&special\cr dynamic-extent&inline&type\cr ftype&notinline&\cr ignorable&optimize&\cr }

An implementation is free to support other (//[[CL:Glossary:implementation-defined]]//) //[[CL:Glossary:declaration identifiers]]// as well.  

A warning might be issued if a //[[CL:Glossary:declaration identifier]]//  is not among those defined above,

is not defined by the //[[CL:Glossary:implementation]]//, is not a //[[CL:Glossary:type]]// //[[CL:Glossary:name]]//,  and has not been declared in a **[[CL:Declarations:declaration]]** //[[CL:Glossary:proclamation]]//.

\beginsubsubsection{Shorthand notation for Type Declarations}

A //[[CL:Glossary:type specifier]]// can be used as a //[[CL:Glossary:declaration identifier]]//. \f{(//type-specifier// \starparam{var})} is taken as shorthand for \f{(type //type-specifier// \starparam{var})}.

\endsubsubsection%{Shorthand notation for Type Declarations}

\endSubsection%{Declaration Identifiers}

\beginSubsection{Declaration Scope} \DefineSection{DeclScope}

//[[CL:Glossary:Declarations]]// can be divided into two kinds: those that apply to the //[[CL:Glossary:bindings]]// of //[[CL:Glossary:variables]]// or //[[CL:Glossary:functions]]//; and those that do not apply to //[[CL:Glossary:bindings]]//.

A //[[CL:Glossary:declaration]]// that appears at the head of a binding //[[CL:Glossary:form]]//  and applies to a //[[CL:Glossary:variable]]// or //[[CL:Glossary:function]]// //[[CL:Glossary:binding]]//  made by that //[[CL:Glossary:form]]// is called a //[[CL:Glossary:bound declaration]]//;  such a //[[CL:Glossary:declaration]]// affects both the //[[CL:Glossary:binding]]// and any references within the //[[CL:Glossary:scope]]// of the //[[CL:Glossary:declaration]]//.  

//[[CL:Glossary:Declarations]]// that are not //[[CL:Glossary:bound declarations]]// are called \newtermidx{free declarations}{free declaration}.

A //[[CL:Glossary:free declaration]]// in a //[[CL:Glossary:form]]// $F1$ that applies to a //[[CL:Glossary:binding]]// for a //[[CL:Glossary:name]]// $N$ //[[CL:Glossary:established]]// by some //[[CL:Glossary:form]]// $F2$ of which $F1$ is a //[[CL:Glossary:subform]]// affects only references to $N$ within $F1$; it does not to apply to other references to $N$ outside of $F1$, nor does it affect the manner in which the //[[CL:Glossary:binding]]// of $N$ by $F2$ is //[[CL:Glossary:established]]//.

//[[CL:Glossary:Declarations]]// that do not apply to //[[CL:Glossary:bindings]]// can only appear  as //[[CL:Glossary:free declarations]]//.

\issue{DECLARATION-SCOPE:NO-HOISTING} \issue{WITH-ADDED-METHODS:DELETE} \issue{GENERIC-FLET-POORLY-DESIGNED:DELETE} The //[[CL:Glossary:scope]]// of a //[[CL:Glossary:bound declaration]]// is the same as the

//[[CL:Glossary:lexical scope]]// of the //[[CL:Glossary:binding]]// to which it applies;

for //[[CL:Glossary:special variables]]//, this means the //[[CL:Glossary:scope]]// that the //[[CL:Glossary:binding]]//  would have had had it been a //[[CL:Glossary:lexical binding]]//.

Unless explicitly stated otherwise, the //[[CL:Glossary:scope]]// of a  //[[CL:Glossary:free declaration]]// includes only the body //[[CL:Glossary:subforms]]// of  the //[[CL:Glossary:form]]// at whose head it appears, and no other //[[CL:Glossary:subforms]]//. The //[[CL:Glossary:scope]]// of //[[CL:Glossary:free declarations]]// specifically does not include //[[CL:Glossary:initialization forms]]// for //[[CL:Glossary:bindings]]// established by the //[[CL:Glossary:form]]// containing the //[[CL:Glossary:declarations]]//.

Some //[[CL:Glossary:iteration forms]]// include step, end-test, or result  //[[CL:Glossary:subforms]]// that are also included in the //[[CL:Glossary:scope]]// of //[[CL:Glossary:declarations]]// that appear in the //[[CL:Glossary:iteration form]]//. Specifically, the //[[CL:Glossary:iteration forms]]// and //[[CL:Glossary:subforms]]// involved are:

\beginlist \item{\bull} \macref{do}, \macref{do*}:  
  //step-forms//, //end-test-form//, and //result-forms//. \item{\bull} \macref{dolist}, \macref{dotimes}:
  //result-form// \item{\bull} \macref{do-all-symbols}, \macref{do-external-symbols}, \macref{do-symbols}:
  //result-form// \endlist

\beginsubsubsection{Examples of Declaration Scope}

Here is an example illustrating the //[[CL:Glossary:scope]]// of //[[CL:Glossary:bound declarations]]//.

\code
 (let ((x 1))                ;[1] 1st occurrence of x
   (declare (special x))     ;[2] 2nd occurrence of x
   (let ((x 2))              ;[3] 3rd occurrence of x
     (let ((old-x x)         ;[4] 4th occurrence of x
           (x 3))            ;[5] 5th occurrence of x
       (declare (special x)) ;[6] 6th occurrence of x
       (list old-x x))))     ;[7] 7th occurrence of x \EV (2 3) \endcode

The first occurrence of \f{x} //[[CL:Glossary:establishes]]// a //[[CL:Glossary:dynamic binding]]// of \f{x} because of the **[[CL:Declarations:special]]** //[[CL:Glossary:declaration]]// for \f{x} in the second line.  The third occurrence of \f{x} //[[CL:Glossary:establishes]]// a //[[CL:Glossary:lexical binding]]// of \f{x} (because there is no **[[CL:Declarations:special]]** //[[CL:Glossary:declaration]]// in the corresponding \macref{let} //[[CL:Glossary:form]]//). The fourth occurrence of \f{x} //[[CL:Glossary:x]]// is a reference to the //[[CL:Glossary:lexical binding]]// of \f{x} established in the third line. The fifth occurrence of \f{x} //[[CL:Glossary:establishes]]// a //[[CL:Glossary:dynamic binding]]// of //[[CL:Glossary:x]]// for the body of the \macref{let} //[[CL:Glossary:form]]// that begins on that line because of the **[[CL:Declarations:special]]** //[[CL:Glossary:declaration]]// for \f{x} in the sixth line. The reference to \f{x} in the fourth line is not affected by the **[[CL:Declarations:special]]** //[[CL:Glossary:declaration]]// in the sixth line  because that reference is not within the ``would-be //[[CL:Glossary:lexical scope]]//'' of the //[[CL:Glossary:variable]]// \f{x} in the fifth line.  The reference to \f{x} in the seventh line is a reference to the //[[CL:Glossary:dynamic binding]]// of //[[CL:Glossary:x]]// //[[CL:Glossary:established]]// in the fifth line.

Here is another example, to illustrate the //[[CL:Glossary:scope]]// of a //[[CL:Glossary:free declaration]]//.  In the following:

\code
 (lambda (&optional (x (foo 1))) ;[1]
   (declare (notinline foo))     ;[2]
   (foo x))                      ;[3] \endcode

the //[[CL:Glossary:call]]// to \f{foo} in the first line might be  compiled inline even though the //[[CL:Glossary:call]]// to \f{foo} in the third line must not be.  This is because the **[[CL:Declarations:notinline]]** //[[CL:Glossary:declaration]]// for \f{foo} in the second line applies only to the body on the third line.  In order to suppress inlining for both //[[CL:Glossary:calls]]//,  one might write:

\code
 (locally (declare (notinline foo)) ;[1]
   (lambda (&optional (x (foo 1)))  ;[2]
     (foo x)))                      ;[3] \endcode

or, alternatively:

\code
 (lambda (&optional                               ;[1]
            (x (locally (declare (notinline foo)) ;[2]
                 (foo 1))))                       ;[3]
   (declare (notinline foo))                      ;[4]
   (foo x))                                       ;[5] \endcode

Finally, here is an example that shows the //[[CL:Glossary:scope]]// of //[[CL:Glossary:declarations]]// in an //[[CL:Glossary:iteration form]]//.

\code
 (let ((x  1))                     ;[1]
   (declare (special x))           ;[2]
     (let ((x 2))                  ;[3]
       (dotimes (i x x)            ;[4]
         (declare (special x)))))  ;[5] \EV 1 \endcode

In this example, the first reference to \f{x} on the fourth line is to the //[[CL:Glossary:lexical binding]]// of \f{x} established on the third line. However, the second occurrence of \f{x} on the fourth line lies within the //[[CL:Glossary:scope]]// of the //[[CL:Glossary:free declaration]]// on the fifth line (because this is the //result-form// of the \macref{dotimes}) and therefore refers to the //[[CL:Glossary:dynamic binding]]// of \f{x}.

\endsubsubsection%{Examples of Declaration Scope}

\endSubsection%{Declaration Scope}
